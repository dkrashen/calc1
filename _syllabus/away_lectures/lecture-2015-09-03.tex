\documentclass[12pt]{amsart}
\usepackage{fullpage}

\newcommand{\course}{MATH 2250}
\newcommand{\semester}{Fall 2015}
\newcommand{\instructor}{Danny Krashen}
\newcommand{\website}{http://dkrashen.github.io/calc1/}

%%%%%%%%%%%%%%%%%%%%%%%%%%%%%%%%%%%%%%%%%%%%%%%%%%%%%%%%%%%%%%%%%%%%%%
% TYPICAL PACKAGES
%%%%%%%%%%%%%%%%%%%%%%%%%%%%%%%%%%%%%%%%%%%%%%%%%%%%%%%%%%%%%%%%%%%%%%

\usepackage{amsmath,amssymb}
\usepackage[all]{xy}
\usepackage{fullpage,enumerate}
\usepackage{url, color, hyperref}

\title{\course, \semester\ (\instructor) \\ Lecture for Thursday 9-3-2015} 

\begin{document}
\maketitle
{\color{blue}The students should have been introduced to the basic
derivative rules, sum rule, constant multiple rule, product rule, quotient
rule, power rule from chapter 3.3. They have not seen yet the chain rule or
derivatives of exponential functions.}

Lecture goal: cover derivatives of exponential and trigonometric functions.
students should practice combining these with the derivative rules which
they already have learned.

\begin{enumerate}[1. ]
\setlength{\itemsep}{.3cm}
\item (15 minutes?)
Reminder of Derivative Rules (Section 3.3)
\begin{enumerate}[ i. ]
\item
Write the product rule, quotient rule and power rules on the board.
\item
Work out the derivative of $f(x) = x^7 + x^3 - 1$, $f(x) = x^3 - 4 \sqrt[3]
x - \frac 1x$, and $\displaystyle f(x) = \frac{x^2 - 9}{\sqrt x + 1}$.
\end{enumerate}

\item (20 minutes?)
Exponential function and its derivative (section 3.3)
\begin{enumerate}[ i. ]
\item
Use the definition of the derivative to work out for $f(x) = a^x$ that
$$\displaystyle f'(x) = a^x \left(\lim_{h \to 0} \frac{a^h -
1}{h}\right).$$
\item
State that the number $e$ is the unique number with the property that

$\lim_{h \to 0} (e^h - 1)/h = 1$, and so $(e^x)' = e^x$.
\item
calculate $f'(x)$ for $f(x) = \frac{e^x + 1}{1 - x^2}$.
\end{enumerate}

\item (20 minutes?)
Derivative of trig functions (section 3.5)
\begin{enumerate}[ i. ]
\item
Use the definition of the derivative, the sum formula to work out $(\sin x)'
= \cos x$ (beginning of section 3.5). students have seen limits of $\sin x
/ x$ and $(\cos x - 1)/x$ as $x \to 0$. 
\item
Show $(\cos x)' = \sin x$. Feel free to skip details, as they have just
seen the basic idea with sine.
\item
calculate $f'(x)$ for $f(x) = 5 e^x + \cos x$, for $f(x) = \sin x \cos x$
and for $\displaystyle f(x) = \frac{\cos x}{1 - \sin x}$. 
\end{enumerate}

\item (20 minutes?)
Group work.

With the time remaining (if any), tell the students to break up into groups
and work on some practice problems like these (write them on the board,
students know well how to break up into groups of about 2-4 students each
and work on problems). They should calculate the derivatives of the
following functions:
\begin{enumerate}[ i. ]
\item
$\tan x$
\item
$\sec x$
\item
$\csc x$
\item
$\cot x$
\item (if they finish the others with time to spare)
$\displaystyle \left(\frac{3x - \tan x}{\sec x + \cos x}\right) e^x \sqrt x$
\end{enumerate}


\end{enumerate}

\end{document}
