\section{Derivatives of the Trigonometric Functions}{}{}
\nobreak
All of the other trigonometric functions can be
expressed in terms of the sine, and so their derivatives can easily be
calculated using the rules we already have. For the cosine we need to
use two identities,
\begin{align*}
\cos x &= \sin(x+{\pi\over2}), \\
\sin x &= -\cos(x+{\pi\over2}). \\
\end{align*}
Now:
\begin{align*}
{d\over dx}\cos x &= {d\over dx}\sin(x+{\pi\over2}) =
\cos(x+{\pi\over2})\cdot 1 = -\sin x \\
{d\over dx}\tan x &= {d\over dx}{\sin x\over \cos x}=
{\cos^2 x + \sin^2 x\over \cos^2 x}={1\over \cos^2 x}=\sec^2 x \\
{d\over dx}\sec x &= {d\over dx}(\cos x)^{-1}=
-1(\cos x)^{-2}(-\sin x) = {\sin x \over \cos^2 x} = \sec x\tan x \\
\end{align*}
The derivatives of the cotangent and cosecant are similar and left as
exercises. 

\begin{exercises}
Find the derivatives of the following functions.

\twocol

\begin{exercise} $\ds \sin x\cos x$
\begin{answer} $\ds \cos^2 x-\sin^2 x$
\end{answer}\end{exercise}

\begin{exercise} $\ds \sin(\cos x)$
\begin{answer} $\ds -\sin x\cos(\cos x)$
\end{answer}\end{exercise}

\begin{exercise} $\ds \sqrt{x\tan x\ }$
\begin{answer} $\ds{\tan x+x\sec^2 x\over2\sqrt{x\tan x\ }}$
\end{answer}\end{exercise}

\begin{exercise} $\ds \tan x/(1+\sin x)$
\begin{answer} $\ds {\sec^2 x(1+\sin x)-\tan x \cos x\over (1+\sin x)^2}$
\end{answer}\end{exercise}

\begin{exercise} $\ds \cot x$
\begin{answer} $\ds  -\csc^2 x$
\end{answer}\end{exercise}

\begin{exercise} $\ds \csc x$
\begin{answer} $\ds  -\csc x\cot x$
\end{answer}\end{exercise}

\begin{exercise} $\ds x^3 \sin (23x^2 )$
\begin{answer} $\ds 3x^2\sin(23x^2)+46x^4\cos(23x^2)$
\end{answer}\end{exercise}

\begin{exercise} $\ds \sin ^2 x + \cos ^2 x$
 \begin{answer} $0$
\end{answer}\end{exercise}

\begin{exercise}  $\ds \sin (\cos (6x) )$
 \begin{answer} $\ds -6\cos(\cos(6x))\sin(6x)$
\end{answer}\end{exercise}

\endtwocol

\begin{exercise} Compute $\ds{d\over d\theta}{\sec \theta\over 1+\sec \theta}$.
 \begin{answer} $\ds \sin\theta/(\cos\theta+1)^2$
\end{answer}\end{exercise}

\begin{exercise} Compute $\ds{d\over dt}t^5 \cos (6t)$.
\begin{answer} $\ds 5t^4\cos(6t)-6t^5\sin(6t)$
\end{answer}\end{exercise}

\begin{exercise} Compute $\ds{d\over dt}{t^3 \sin (3t)\over\cos (2t)}$.
\begin{answer} $\ds 3t^2(\sin(3t)+t\cos(3t))/\cos(2t)+2t^3\sin(3t)\sin(2t)/\cos^2(2t)$
\end{answer}\end{exercise}

\begin{exercise} Find all points on the graph of
$\ds f(x)=\sin^2(x)$ at which the tangent line is horizontal.
\begin{answer} $n\pi/2$, any integer $n$
\end{answer}\end{exercise}

\begin{exercise} Find all points on the graph of $\ds f(x) = 2\sin(x) -
\sin^2(x)$ at which the tangent line is horizontal.
\begin{answer} $\pi/2+n\pi$, any integer $n$
\end{answer}\end{exercise}

\begin{exercise} Find an
 equation for the tangent line to $\ds \sin^2(x)$ at 
$x=\pi/3$.
\begin{answer} $\sqrt3x/2+3/4-\sqrt3\pi/6$
\end{answer}\end{exercise}

\begin{exercise} Find an equation for the tangent line to $\ds \sec ^2 x$
at $x=\pi/3$.
\begin{answer} $\ds 8\sqrt3x+4-8\sqrt3\pi/3$
\end{answer}\end{exercise}

\begin{exercise} Find an equation for the tangent line to $\ds \cos ^2 x -
\sin ^2 (4x)$ at $x=\pi/6$.
\begin{answer} $\ds 3\sqrt3x/2-\sqrt3\pi/4$
\end{answer}\end{exercise}

\begin{exercise} Find the points on the curve $\ds y= x+ 2\cos x$ that have a
horizontal tangent line.
\begin{answer} $\ds \pi/6+2n\pi$, $5\pi/6+2n\pi$, any integer $n$
\end{answer}\end{exercise}

\begin{exercise} Let $C$ be a circle of radius $r$. Let $A$ be an arc on $C$
subtending a central angle $\theta$. Let $B$ be the chord of
$C$ whose endpoints are the endpoints of $A$. (Hence, $B$ also
subtends $\theta$.) Let $s$ be the length of $A$
and let $d$ be the length of $B$. Sketch a diagram of the situation
and compute $\ds \lim_{\theta \to 0^+ } s/d$.
\end{exercise}

\end{exercises}

