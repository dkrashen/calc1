\section{Asymptotes and Other Things to Look For}  {}{}
\nobreak
A vertical asymptote\index{asymptote} is a place where the function
becomes infinite, typically because the formula for the function has a
denominator that becomes zero.  For example, the reciprocal function
$f(x)=1/x$ has a vertical asymptote at $x=0$, and the function $\tan
x$ has a vertical asymptote at $x=\pi/2$ (and also at $x=-\pi/2$,
$x=3\pi/2$, etc.).  Whenever the formula for a function contains a
denominator it is worth looking for a vertical asymptote by 
checking to see if the denominator can ever be zero, and then checking
the limit at such points. Note that there is not always a vertical
asymptote where the derivative is zero: $f(x)=(\sin x)/x$ has a zero
denominator at $x=0$, but since $\ds \lim_{x\to 0}(\sin x)/x=1$ there is
no asymptote there.

A horizontal asymptote is a horizontal line to which $f(x)$ gets closer and
closer as $x$ approaches $\infty$ (or as $x$ approaches $-\infty$).  For
example, the reciprocal function has the $x$-axis for a horizontal
asymptote.  Horizontal asymptotes can be identified by computing 
the limits $\ds \lim_{x \to \infty}f(x)$ and $\ds \lim_{x \to -\infty}f(x)$.
Since $\ds \lim_{x \to \infty}1/x=\lim_{x \to -\infty}1/x=0$, the line
$y=0$ (that is, the $x$-axis) is a horizontal asymptote in both directions.

Some functions have asymptotes that are neither horizontal nor
vertical, but some other line. Such asymptotes are somewhat more
difficult to identify and we will ignore them.

If the domain of the function does not extend out to infinity, we should
also ask what happens as $x$ approaches the boundary of the domain.  For
example, the function $\ds y=f(x)=1/\sqrt{r^2-x^2}$ has domain $-r<x<r$, and
$y$ becomes infinite as $x$ approaches either $r$ or $-r$. In this
case we might also identify this behavior because when $x=\pm r$ the
denominator of the function is zero.

If there are any points where the derivative fails to exist (a cusp or
corner), then we should take special note of what the function does at such
a point.

Finally, it is worthwhile to notice any symmetry.  A function $f(x)$ that
has the same value for $-x$ as for $x$, i.e., $f(-x)=f(x)$, is called an
``even function.''  Its graph is symmetric with respect to the $y$-axis.
Some examples of even functions are: $\ds x^n$ when $n$ is an even number,
$\cos x$, and $\ds \sin^2x$.  On the other hand, a function that satisfies the
property $f(-x)=-f(x)$ is called an ``odd function.''  Its graph is
symmetric with respect to the origin.  Some examples of odd functions are:
$x^n$ when $n$ is an odd number, $\sin x$, and $\tan x$.  Of course, most
functions are neither even nor odd, and do not have any particular
symmetry.
 
\begin{exercises}

Sketch the curves. Identify clearly any interesting features, including
local maximum and minimum points, inflection points, asymptotes, and
intercepts. 

\twocol

\begin{exercise} $\ds y=x^5-5x^4+5x^3$
\end{exercise}

\begin{exercise} $\ds y=x^3-3x^2-9x+5$
\end{exercise}

\begin{exercise} $\ds y=(x-1)^2(x+3)^{2/3}$
\end{exercise}

\begin{exercise} $\ds x^2+x^2y^2=a^2y^2$, $a>0$.
\end{exercise}

\begin{exercise} $\ds y = 4x+\sqrt{1-x}$
\end{exercise}

\begin{exercise} $\ds y = (x+1)/\sqrt{5x^2 + 35}$
\end{exercise}

\begin{exercise} $\ds y= x^5 - x$
\end{exercise}

\begin{exercise} $\ds y = 6x + \sin 3x$
\end{exercise}

\begin{exercise} $\ds y = x+ 1/x$
\end{exercise}

\begin{exercise} $\ds y = x^2+ 1/x$
\end{exercise}

\begin{exercise} $\ds y = (x+5)^{1/4}$
\end{exercise}

\begin{exercise} $\ds y = \tan^2 x$
\end{exercise}

\begin{exercise} $\ds y =\cos^2 x - \sin^2 x$
\end{exercise}

\begin{exercise} $\ds y = \sin^3 x$
\end{exercise}

\begin{exercise} $\ds y=x(x^2+1)$
\end{exercise}

\begin{exercise} $\ds y=x^3+6x^2 + 9x$
\end{exercise}

\begin{exercise} $\ds y=x/(x^2-9)$
\end{exercise}

\begin{exercise} $\ds y=x^2/(x^2+9)$
\end{exercise}

\begin{exercise} $\ds y=2\sqrt{x} - x$
\end{exercise}

\begin{exercise} $\ds y=3\sin(x) - \sin^3(x)$, for $x\in[0,2\pi]$
\end{exercise}

\begin{exercise} $\ds y=(x-1)/(x^2)$
\end{exercise}

\endtwocol

\msk
\noindent
For each of the following five functions, identify any vertical and horizontal
asymptotes, and identify intervals on which the function is 
concave up and increasing; concave up and decreasing; concave
down and increasing; concave down and decreasing.

\begin{exercise} $f(\theta)=\sec(\theta)$ \end{exercise}
\begin{exercise} $\ds f(x) = 1/(1+x^2)$\end{exercise}
\begin{exercise} $\ds f(x) = (x-3)/(2x-2)$ \end{exercise}
\begin{exercise} $\ds f(x) = 1/(1-x^2)$\end{exercise}
\begin{exercise} $\ds f(x) = 1+1/(x^2)$\end{exercise}

\begin{exercise} Let $\ds f(x) = 1/(x^2-a^2)$, where $a\geq0$.  Find any
 vertical and horizontal asymptotes and the intervals upon which the
 given function is concave up and increasing; concave up and
 decreasing; concave down and increasing; concave down and decreasing.
 Discuss how the value of $a$ affects these features.
\end{exercise}



\end{exercises}

\endinput

% Example 1
\begin{example}

Sketch $y=x^3-x$.

\smallskip
\begin{rightindent}{3.5in}
First, we set $0=y'=3x^2-1$, which has solutions $x=\pm\sqrt{3}/3=\pm
0.577$.  The corresponding $y$-coordinates are $-0.385$ and $+0.385$, i.e.,
the two ``critical points'' are $(0.577,-0.385)$ and $(-0.577,0.385)$.  The
second derivative test gives $y''=6x$, which is positive for the first
point and negative for the second.  Thus, the first of the two points (in
the fourth quadrant) is a local minimum, and the second is a local maximum.
Since $y''>0$ when $x>0$ and $y''<0$ when $x<0$, it follows that the curve
is concave upward in the right half of the graph and concave down-
\end{rightindent}
\hfill
%--- figure ND (curve sketching--Example 1)--------------------
\begin{psfigure}{1.95in}{2.0in}{124.figND.ps}
\end{psfigure}
%------------------------------
ward in the left half. The two halves are
separated by the inflection point at the origin.  This curve has no
asymptotes.  It does have symmetry, however, because it is an odd function.
Its graph is shown above.

\end{example}

\begin{rightindent}{3in}
% Example 2
\begin{example}

Sketch $y=x^3+x$.

\smallskip

The only difference with Example 13.1 is the $+$ in front of the $x$.  But
this means that the derivative $y'=3x^2+1$ is {\it never} zero, and hence
there are no maxima or minima.  In fact, the function is always increasing,
because $y'$ is always positive.  The second derivative $y''=6x$ is the
same as in the last problem, and hence the concavity situation is the same.
In particular, this curve also has an inflection point at the origin.

\end{example}
\end{rightindent}
\hfill
%--- figure NE (curve sketching--Example 2)--------------------
\begin{psfigure}{2in}{2in}{124.figNE.ps}
\end{psfigure} 
%------------------------------

% Example 3
\begin{example}

Sketch $y=x^2-\cos(2x)$ for $-\pi/2\le x\le\pi/2$.

\smallskip

When we set $0=y'=2x+2\sin(2x)$, we obtain the equality $\sin(2x)=-x$.
However, from a quick sketch of the two curves $\sin(2x)$ and $-x$, we
immediately see that the only $x$ for which they are equal is $x=0$.  When
$x=0$ the $y$-coordinate is $0^2-\cos(2\cdot 0)=-1$, so our critical point
is $(0,-1)$.  Since $y''=2+4\cos(2x)$, which is positive when $x=0$, the
second derivative test tells us that $(0,1)$ is a local minimum.  To find
inflection points we set $0=y''=2+4\cos(2x)$.  This gives $\cos(2x)=-0.5$.
Looking at our table in the section on trig functions, we see that in the
range from $x=0$ to $x=\pi/2$ the equality $\cos(2x)=-0.5$ holds when
$2x=\frac{2}{ 3}\pi$, 
\begin{rightindent}{3in}
\noindent
i.e., $x=\pi/3$.  Since $\cos(-2x)=\cos(2x)$, the
equality also holds when $x=-\pi/3$.  Thus, the points $(\pi/3,1.5966)$ and
$(-\pi/3,1.5966)$ are inflection points.  Between these two inflection
points the second derivative is positive (concave up), whereas for
$x>\pi/3$ and for $x<-\pi/3$ the second derivative is negative (concave
downward).  Finally, note that $x^2-\cos(2x)$ is an even function, and so
the graph is symmetrical with respect to the $y$-axis.
\end{rightindent}
%--- figure NF (curve sketching--Example 3)--------------------
\begin{psfigure}{2.0in}{1.8in}{124.figNF.ps}
\end{psfigure}
%------------------------------
\end{example}

% Example 4
\begin{example}

Sketch $y=x^2+\frac{1}{ x}$.

\smallskip

Setting $0=y'=2x-\frac{1}{ x^2}$, we solve this by bringing $1/x^2$ to the
left and clearing denominators: $1=2x\cdot x^2=2x^3$.  So $x={\root
3\of{0.5}}= 0.7937$.  The corresponding $y$-coordinate is 1.8899.  Using
the second derivative $y''=2+2\cdot x^{-3}$, we see that this 
\begin{rightindent}{3in}
\noindent
is a local minimum.  To find inflections, we set $2+2/x^3=0$.  Clearing
denominators and solving for $x$ gives $x^3=-1$, and so $x=-1$.  Thus, the
point $(-1,0)$ is an inflection.  In this example we have an asymptote when
$x=0$.  To the right of the asymptote (i.e., for positive $x$), the second
derivative is always positive; whereas for negative $x$ the second
derivative is positive when $x<-1$ and negative when $x$ is between $-1$
and $0$.  Thus, the interval $-1<x<0$ is a region of downward concavity;
the graph is concave upward outside of this interval.  Putting all this
together leads to the graph at the right.
\end{rightindent}
\hfill
%--- figure NG (curve sketching--Example 4)--------------------
%
\begin{psfigure}{2.0in}{2.0in}{124.figNG.ps}
\end{psfigure}
%------------------------------

\smallskip

There is another way to think of this example.  Our function is the sum of
two functions $x^2$ and $1/x$.  The former function is by far the larger of
the two when $x$ is large positive or large negative, whereas the
reciprocal function is by far the more important when $x$ is near 0.  Thus,
the graph resembles $1/x$ when $x$ is near 0 and resembles $x^2$ when $x$
is far from 0.  Roughly speaking, the inflection point $(-1,0)$ and the
local minimum $(0.7937,1.8899)$ mark the transition from behaving like the
graph of $x^2$ to behaving like the graph of $1/x$.

\end{example}

% Example 5
\begin{example}

Sketch (a) $y=x^3$ and (b) $y=x^4$.

\smallskip


(a) Setting $0=y'=3x^2$, we see that the origin is a possible maximum or
minimum.  However, the second derivative test tells us nothing, since
$y''=6x$ also is zero when $x=0$.  In fact, even though $y'=0$ when $x=0$,
the origin is neither a maximum nor a minimum.  Rather, it is a point of
inflection, separating the concave downward region in the third quadrant
from the concave upward region in the first quadrant.

\smallskip


(b) Again we see that both the first derivative and the second derivative
vanish at the origin (and neither derivative is zero anywhere else).  This
time, however, the origin is a local minimum.  Even though the second derivative
test doesn't tell us this, we can see directly that, since $x^4$ is
positive for nonzero $x$, its smallest possible value is when $x=0$.  Note
that the origin is {\it not} a point of inflection, even though $y''=0$
there.  This is because $y''=12x^2>0$ both for $x>0$ and for $x<0$, so
everywhere we have upward concavity.  (It is rare for a point where $y''=0$
not to be an inflection point; this can occur only when the {\it third}
derivative $y^{\prime\prime\prime}$ is also zero at the same point.)

\begin{centering}
%--------------------------------------------------------------
%--- figure NH (curve sketching--Example 5)--------------------
\begin{psfigure}{4.5in}{1.5in}{124.figNH.ps}
\end{psfigure}\\
%------------------------------
\end{centering}

\end{example}


% Example 6
\begin{example}

Sketch $y=x^5-5x^4+5x^3$.

\smallskip

First, we set $0=y'=5x^4-20x^3+15x^2$.  To solve this, we factor out what
we can, namely $5x^2$.  This leaves a quadratic that can be factored either
by inspection or by the quadratic formula.  The result is
$0=y'=5x^2(x-1)(x-3)$.  Thus, the critical points are $(0,0)$, $(1,1)$, and
$(3,-27)$.  Using $y''=20x^3-60x^2+30x=10x(2x^2-6x+3)$, we see from the
second derivative test that $(1,1)$ is a local maximum and $(3,-27)$ is a
local minimum, but we get no information about $(0,0)$.  Setting $0=y''$
and using the quadratic formula to find the roots of $2x^2-6x+3$, we find
the following three points of inflection: $(0,0)$, $(0.634,0.569)$,
$(2.366,-16.32)$.

\begin{rightindent}{3.0in}
\noindent
  In a complicated case like
this, it is also worthwhile to see what the function is doing when $x$ is
large positive or large negative.  If $x$ is large, the $x^5$ term in our
function dominates (is greater in absolute value than all the other terms).
Thus, the function heads upward steeply into the first quadrant as
$x\longrightarrow+\infty$, and it heads steeply down into the third
quadrant as $x\longrightarrow-\infty$.  Putting this information together,
we obtain the graph shown above.  The curve is concave downward in the
third quadrant, and also between the two points of inflection
$(0.634,0.569)$ and $(2.366,-16.32)$.  For $0<x<0.634$ and for $x>2.366$
the curve is concave upward.
\end{rightindent}
\hfill
%--- figure NI (curve sketching--Example 6)--------------------
%
\begin{psfigure}{2.45in}{2.5in}{124.figNI.ps}
\end{psfigure}
%------------------------------

\end{example}

\newpage

%
% Homework on curve sketching
%
\begin{homework}\
\label{homework:13}

\medskip

\begin{exercise} %1.
Suppose that $n$ is an integer greater than 2.  On the curve $y=f(x)=x^n$,
what sort of point is the origin?  Sketch the curve, and indicate the
concavity.

\smallskip

\begin{exercise} %2.
In parts (a)-(j) below, find each max/min point (using the second
derivative test to be sure what type of point it is), point of inflection,
and asymptote (vertical, horizontal, or slanted), if there are any.  Also
indicate where the curve is concave up and concave down.  Sketch the graph.

\noindent
(a) $y=x^2-x$, (b) $y=2+3x-x^3$, (c) $y=x^3-9x^2+24x$,
(d) $y=x^4-2x^2+3$, (e) $y=3x^4-4x^3$,
(f) $y=(x^2-1)/x$,
(g) $y=3x^2-(1/x^2)$, (h) $y=\cos(2x)-x$, (i) $y=\sin x+\cos x$,
(j) $y=\tan(x/2)-x$.

\smallskip

\begin{exercise} %3.
Suppose  $\begin{cases} f'(x) > 0& \text{for $|x| > 2$;}\\
                                  f'(x) < 0& \text{for $|x| < 2$;}\\
                                  f''(x) < 0& \text{for $x  < 0$; and }\\
                                  f''(x) > 0& \text{for  $x  > 0$}\end{cases}$.

\noindent
From the given information, sketch a possible graph of $f(x)$.
How could your answer vary and still be correct?


\smallskip

\begin{exercise} %4.
Suppose $f''(x) <0$ for $x<1$ and  $f''(x)>0$ for $x>1$.
Suppose also that $f(1)=1$.  Sketch a possible graph of $f(x)$, assuming that

         i) $f'(1) = 0$,   ii) $f'(1) > 0$ and  iii) $f'(1) < 0$.

\smallskip

\begin{rightindent}{3.4in}
\begin{exercise} %5.
At the right is a sketch of $y=x^6-5x^4$.

(a) Find the exact coordinates $(x,y)$ of the local maxima and local
minima.

(b) For what values of $x$ is the function concave upward?
									
(c) Find the exact $x$-coordinates of all points of inflection. 
({\bf Hint:} Color the part of the curve that is concave up blue and color
the part that is concave down red. Points of inflection occur {\it only}
where the curve {\it changes} color!)
\end{rightindent}
\hfill
\begin{psfigure}{2.0in}{2.0in}{124.figNJ.ps}
\end{psfigure}

(d) Explain how you {\it know from your calculations} (not from the
sketch you were given) which of your answers to part (a) are minima and
which are maxima.


\end{homework}


%-----------------end of Chapter 13 (curve sketching) -----



