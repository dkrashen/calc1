\section{Slopes in polar coordinates}{}{}
\nobreak
When we describe a curve using polar coordinates, it is still a curve
in the $x$-$y$ plane. We would like to be able to compute slopes and
areas for these curves using polar coordinates.

We have seen that $x=r\cos\theta$ and $y=r\sin\theta$ describe the
relationship between polar and rectangular coordinates. If in turn we
are interested in a curve given by $r=f(\theta)$, then we can write
$x=f(\theta)\cos\theta$ and $y=f(\theta)\sin\theta$, describing $x$
and $y$ in terms of $\theta$ alone. The first of these equations
describes $\theta$ implicitly in terms of $x$, so using the chain rule
we may compute
$${dy\over dx}={dy\over d\theta}{d\theta\over dx}.$$
Since $d\theta/dx=1/(dx/d\theta)$, we can instead compute
$$
  {dy\over dx}={dy/d\theta\over dx/d\theta}=
  {f(\theta)\cos\theta + f'(\theta)\sin\theta\over
  -f(\theta)\sin\theta + f'(\theta)\cos\theta}.
$$

\begin{example} Find the points at which the curve given by
$r=1+\cos\theta$ has a vertical or horizontal tangent line. Since this
function has period $2\pi$, we may restrict our attention to the
interval $[0,2\pi)$ or $(-\pi,\pi]$, as convenience dictates.
First, we compute the slope:
$$
  {dy\over dx}={(1+\cos\theta)\cos\theta-\sin\theta\sin\theta\over
    -(1+\cos\theta)\sin\theta-\sin\theta\cos\theta}=
  {\cos\theta+\cos^2\theta-\sin^2\theta\over
    -\sin\theta-2\sin\theta\cos\theta}.
$$
This fraction is zero when the numerator is zero (and the denominator
is not zero). The numerator is $\ds \ds 2\cos^2\theta+\cos\theta-1$ so by the
quadratic formula
$$
  \cos\theta={-1\pm\sqrt{1+4\cdot2}\over 4} = -1 
  \quad\hbox{or}\quad {1\over 2}.
$$
This means $\theta$ is $\pi$ or $\pm \pi/3$.
However, when $\theta=\pi$, the denominator is
also $0$, so we cannot conclude that the tangent line is horizontal. 

Setting the denominator to zero we get
$$\eqalign{
  -\theta-2\sin\theta\cos\theta &= 0 \\
  \sin\theta(1+2\cos\theta)&=0, \\}
$$ 
so either $\sin\theta=0$ or $\cos\theta=-1/2$. The first is true when
$\theta$ is $0$ or $\pi$, the second when $\theta$ is $2\pi/3$ or
$4\pi/3$. However, as above, when $\theta=\pi$, the numerator is also $0$, so we
cannot conclude that the tangent line is
vertical. Figure~\xrefn{fig:cardioid tangents} shows points
corresponding to $\theta$ equal to $0$, $\pm 1.318$, $2\pi/3$ and
$4\pi/3$ on the graph of the function. Note that when $\theta=\pi$ the
curve hits the origin and does not have a tangent line.
\end{example}


\figure
\vbox{\beginpicture
\normalgraphs
\ninepoint
\setcoordinatesystem units <12truemm,12truemm>
\setplotarea x from -1 to 2.5, y from -1.5 to 1.5
\axis left shiftedto x=0 /
\axis bottom shiftedto y=0 /
\multiput {$\bullet$} at 2 0 0.75 1.299 0.75 -1.299 -0.25 0.433  -0.25 -0.433 /
\setquadratic
\plot 2.000 0.000 1.984 0.208 1.935 0.411 1.856 0.603 1.748 0.778 
1.616 0.933 1.464 1.063 1.295 1.166 1.117 1.240 0.933 1.285 
0.750 1.299 0.572 1.285 0.405 1.245 0.251 1.182 0.115 1.098 
-0.000 1.000 -0.094 0.891 -0.165 0.775 -0.214 0.657 -0.241 0.542 
-0.250 0.433 -0.242 0.333 -0.221 0.246 -0.191 0.172 -0.155 0.112 
-0.116 0.067 -0.079 0.035 -0.047 0.015 -0.021 0.005 -0.005 0.001 
0.000 0.000 -0.005 -0.001 -0.021 -0.005 -0.047 -0.015 -0.079 -0.035 
-0.116 -0.067 -0.155 -0.112 -0.191 -0.172 -0.221 -0.246 -0.242 -0.333 
-0.250 -0.433 -0.241 -0.542 -0.214 -0.657 -0.165 -0.775 -0.094 -0.891 
0.000 -1.000 0.115 -1.098 0.251 -1.182 0.405 -1.245 0.572 -1.285 
0.750 -1.299 0.933 -1.285 1.117 -1.240 1.295 -1.166 1.464 -1.063 
1.616 -0.933 1.748 -0.778 1.856 -0.603 1.935 -0.411 1.984 -0.208 
2.000 0.000 /
\endpicture}
\figrdef{fig:cardioid tangents}
\endfigure{Points of vertical and horizontal tangency for $r=1+\cos\theta$.}

We know that the second derivative $f''(x)$ is useful in describing
functions, namely, in describing concavity. We can compute $f''(x)$ in
terms of polar coordinates as well. We already know how to write 
$dy/dx=y'$ in terms of $\theta$, then
$$
  {d\over dx}{dy\over dx}= {dy'\over dx}={dy'\over
    d\theta}{d\theta\over dx}={dy'/d\theta\over dx/d\theta}.
$$

\begin{example} We find the second derivative for the cardioid
$r=1+\cos\theta$:
$$\eqalign{
  {d\over d\theta}{\cos\theta+\cos^2\theta-\sin^2\theta\over
  -\sin\theta-2\sin\theta\cos\theta}\cdot{1\over dx/d\theta} &=\cdots=
  {3(1+\cos\theta)\over (\sin\theta+2\sin\theta\cos\theta)^2}
  \cdot{1\over-(\sin\theta+2\sin\theta\cos\theta)} \\
  &={-3(1+\cos\theta)\over(\sin\theta+2\sin\theta\cos\theta)^3}. \\}
$$
The ellipsis here represents rather a substantial amount of algebra.
We know from above that the cardioid has horizontal tangents at $\pm
\pi/3$; substituting these values into the second derivative we get
$\ds y''(\pi/3)=-\sqrt{3}/2$ and $\ds y''(-\pi/3)=\sqrt{3}/2$,
indicating concave down and concave up respectively. This agrees with
the graph of the function.
\end{example}

\begin{exercises}
%Keisler

\noindent Compute $y'=dy/dx$ and $\ds y''=d^2y/dx^2$.

\twocol

\begin{exercise} $r=\theta$
\begin{answer}$(\theta\cos\theta+\sin\theta)/(-\theta\sin\theta+\cos\theta)$,
$\ds (\theta^2+2)/(-\theta\sin\theta+\cos\theta)^3$
\end{answer}\end{exercise}

\begin{exercise} $r=1+\sin\theta$
\begin{answer} $\ds {\cos\theta+2\sin\theta\cos\theta\over
\cos^2\theta-\sin^2\theta-\sin\theta}$,
$\ds {3(1+\sin\theta)\over(\cos^2\theta-\sin^2\theta-\sin\theta)^3}$
\end{answer}\end{exercise}

\begin{exercise} $r=\cos\theta$
\begin{answer} $\ds (\sin^2\theta-\cos^2\theta)/(2\sin\theta\cos\theta)$,
$\ds -1/(4\sin^3\theta\cos^3\theta)$
\end{answer}\end{exercise}

\begin{exercise} $r=\sin\theta$
\begin{answer} $\ds {2\sin\theta\cos\theta\over\cos^2\theta-\sin^2\theta}$,
$\ds {2\over(\cos^2\theta-\sin^2\theta)^3}$
\end{answer}\end{exercise}

\begin{exercise} $r=\sec\theta$
\begin{answer} undefined
\end{answer}\end{exercise}

\begin{exercise} $r=\sin(2\theta)$
\begin{answer} $\ds {2\sin\theta-3\sin^3\theta\over3\cos^3\theta-2\cos\theta}$,
$\ds {3\cos^4\theta-3\cos^2\theta+2\over2\cos^3\theta(3\cos^2\theta-2)^3}$
\end{answer}\end{exercise}

\endtwocol

\bigbreak
\noindent Sketch the curves over the interval $[0,2\pi]$ unless
otherwise stated.

\twocol

\begin{exercise} $r=\sin\theta+\cos\theta$

\begin{exercise} $r=2+2\sin\theta$

\begin{exercise} $\ds r={3\over2}+\sin\theta$

\begin{exercise} $r= 2+\cos\theta$

\begin{exercise} $\ds r={1\over2}+\cos\theta$

\begin{exercise} $\ds r=\cos(\theta/2), 0\le\theta\le4\pi$

\begin{exercise} $r=\sin(\theta/3), 0\le\theta\le6\pi$

\begin{exercise} $\ds r=\sin^2\theta$

\begin{exercise} $\ds r=1+\cos^2(2\theta)$

\begin{exercise} $\ds r=\sin^2(3\theta)$

\begin{exercise} $\ds r=\tan\theta$

\begin{exercise} $\ds r=\sec(\theta/2), 0\le\theta\le4\pi$

\begin{exercise} $\ds r=1+\sec\theta$

\begin{exercise} $\ds r={1\over 1-\cos\theta}$

\begin{exercise} $\ds r={1\over 1+\sin\theta}$

\begin{exercise} $\ds r=\cot(2\theta)$

\begin{exercise} $\ds r=\pi/\theta, 0\le\theta\le\infty$

\begin{exercise} $\ds r=1+\pi/\theta, 0\le\theta\le\infty$

\begin{exercise} $\ds r=\sqrt{\pi/\theta}, 0\le\theta\le\infty$

\endtwocol

\end{exercises}

