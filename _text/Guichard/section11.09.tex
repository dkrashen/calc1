\section{Calculus with Power Series}{}{}
\nobreak
Now we know that some functions can be expressed as power series,
which look like infinite polynomials. Since calculus, that is,
computation of derivatives and antiderivatives, is easy for
polynomials, the obvious question is whether the same is true
for infinite series. The answer is yes:

\begin{theorem} Suppose the power series $f(x)=\ds\sum_{n=0}^\infty a_n(x-a)^n$ has
radius of convergence $R$. Then
$$\eqalign{
  f'(x)&=\sum_{n=0}^\infty na_n(x-a)^{n-1}, \\
  \int f(x)\,dx &= C+\sum_{n=0}^\infty {a_n\over n+1}(x-a)^{n+1}, \\
}$$
and these two series have radius of convergence $R$ as well.
\end{proof}

\begin{example}
Starting with the geometric series:
$$\eqalign{
  {1\over 1-x} &= \sum_{n=0}^\infty x^n \\
  \int{1\over 1-x}\,dx &= -\ln|1-x| = \sum_{n=0}^\infty {1\over
    n+1}x^{n+1} \\
  \ln|1-x| &= \sum_{n=0}^\infty -{1\over n+1}x^{n+1} \\
}$$
when $|x|<1$. The series does not converge when $x=1$ but does
converge when $x=-1$ or $1-x=2$. The interval of convergence is
$[-1,1)$, or $0<1-x\le2$, so
we can use the series to represent $\ln(x)$
when $0<x\le2$. For example
$$
  \ln(3/2)=\ln(1--1/2)=
  \sum_{n=0}^\infty (-1)^n{1\over n+1}{1\over 2^{n+1}}
$$
and so
$$
  \ln(3/2)\approx {1\over 2}-{1\over 8}+{1\over 24}-{1\over 64}
  +{1\over 160}-{1\over 384}+{1\over 896}
  ={909\over 2240}\approx 0.406
.$$
Because this is an alternating series with decreasing terms,
we know that the true value is between $909/2240$ and
$909/2240-1/2048=29053/71680\approx .4053$, so correct to two decimal
places the value is $0.41$. 

What about $\ln(9/4)$? Since $9/4$ is larger than 2 we cannot use the
series directly, but
$$\ln(9/4)=\ln((3/2)^2)=2\ln(3/2)\approx 0.82,$$
so in fact we get a lot more from this one calculation than first
meets the eye. To estimate the true value accurately we actually need
to be a bit more careful.
When we multiply by two we know that the true value is between
$0.8106$ and $0.812$, so rounded to two decimal places the true value
is $0.81$.
\end{example}

\begin{exercises}

\begin{exercise} Find a series representation for $\ln 2$.
\begin{answer} the alternating harmonic series
\end{answer}\end{exercise}

\begin{exercise} Find a power series representation for $\ds 1/(1-x)^2$.
\begin{answer} $\ds\sum_{n=0}^\infty (n+1)x^n$
\end{answer}\end{exercise}

\begin{exercise} Find a power series representation for $\ds 2/(1-x)^3$.
\begin{answer} $\ds\sum_{n=0}^\infty (n+1)(n+2)x^n$
\end{answer}\end{exercise}

\begin{exercise} Find a power series representation for $\ds 1/(1-x)^3$.
What is the radius of convergence?
\begin{answer} $\ds\sum_{n=0}^\infty {(n+1)(n+2)\over 2}x^n$, $R=1$
\end{answer}\end{exercise}

\begin{exercise} Find a power series representation for $\ds\int\ln(1-x)\,dx$.
\begin{answer} $\ds C+\sum_{n=0}^\infty {-1\over (n+1)(n+2)}x^{n+2}$ 
\end{answer}\end{exercise}

\end{exercises}

