\section{Taylor Series}{}{}
\nobreak
We have seen that some functions can be represented as series, which
may give valuable information about the function. So far, we have seen
only those examples that result from manipulation of our one
fundamental example, the geometric series. We would like to start with
a given function and produce a series to represent it, if possible.

Suppose that $\ds f(x)=\sum_{n=0}^\infty a_nx^n$ on some interval of
convergence. Then we know that we can compute derivatives of $f$ by
taking derivatives of the terms of the series. Let's look at the first
few in general:
$$\eqalign{
  f'(x)&=\sum_{n=1}^\infty n a_n x^{n-1}=a_1 + 2a_2x+3a_3x^2+4a_4x^3+\cdots \\
  f''(x)&=\sum_{n=2}^\infty n(n-1) a_n x^{n-2}=2a_2+3\cdot2a_3x
    +4\cdot3a_4x^2+\cdots \\
  f'''(x)&=\sum_{n=3}^\infty n(n-1)(n-2) a_n x^{n-3}=3\cdot2a_3
    +4\cdot3\cdot2a_4x+\cdots \\
}$$
By examining these it's not hard to discern the general pattern. The
$k$th derivative must be
$$\eqalign{
  f^{(k)}(x)&=\sum_{n=k}^\infty n(n-1)(n-2)\cdots(n-k+1)a_nx^{n-k} \\
  &=k(k-1)(k-2)\cdots(2)(1)a_k+(k+1)(k)\cdots(2)a_{k+1}x+{} \\
  &\qquad {}+(k+2)(k+1)\cdots(3)a_{k+2}x^2+\cdots \\
}$$
We can shrink this quite a bit by using factorial notation:
$$
  f^{(k)}(x)=\sum_{n=k}^\infty {n!\over (n-k)!}a_nx^{n-k}=
  k!a_k+(k+1)!a_{k+1}x+{(k+2)!\over 2!}a_{k+2}x^2+\cdots
$$
Now substitute $x=0$:
$$f^{(k)}(0)=k!a_k+\sum_{n=k+1}^\infty {n!\over (n-k)!}a_n0^{n-k}=k!a_k,$$
and solve for $\ds a_k$:
$$a_k={f^{(k)}(0)\over k!}.$$
Note the special case, obtained from the series for $f$ itself, that
gives $\ds f(0)=a_0$.

So if a function $f$ can be represented by a series, we know just what
series it is. Given a function $f$, the series
$$\sum_{n=0}^\infty {f^{(n)}(0)\over n!}x^n$$
is called the {\dfont Maclaurin 
series\index{Maclaurin series}\index{series!Maclaurin}\/} for $f$.

\begin{example} Find the Maclaurin series for $f(x)=1/(1-x)$. We need to
compute the derivatives of $f$ (and hope to spot a pattern).
$$\eqalign{
  f(x)&=(1-x)^{-1} \\
  f'(x)&=(1-x)^{-2} \\
  f''(x)&=2(1-x)^{-3} \\
  f'''(x)&=6(1-x)^{-4} \\
  f^{(4)}&=4!(1-x)^{-5} \\
  &\vdots \\
  f^{(n)}&=n!(1-x)^{-n-1} \\
}$$
So
$${f^{(n)}(0)\over n!}={n!(1-0)^{-n-1}\over n!}=1$$
and the Maclaurin series is
$$\sum_{n=0}^\infty 1\cdot x^n=\sum_{n=0}^\infty x^n,$$
the geometric series.
\end{example}

A warning is in order here. Given a function $f$ we may be able to
compute the Maclaurin series, but that does not mean we have found a
series representation for $f$. We still need to know where the series
converges, and if, where it converges, it converges to $f(x)$. While
for most commonly encountered functions the Maclaurin series does
indeed converge to $f$ on some interval, this is not true of all
functions, so care is required.

As a practical matter, if we are interested in using a series to
approximate a function, we will need some finite number of terms of
the series. Even for functions with messy derivatives we can compute
these using computer software like Sage. If we want to
know the whole series, that is, a typical term in the series, we need
a function whose derivatives fall into a pattern that we can
discern. A few of the most important functions are fortunately very
easy.

\begin{example} Find the Maclaurin series for $\sin x$.

The derivatives are quite easy: $f'(x)=\cos x$, $f''(x)=-\sin x$,
$f'''(x)=-\cos x$, $\ds f^{(4)}(x)=\sin x$, and then the pattern
repeats. We want to know the derivatives at zero:
1, 0, $-1$, 0, 1, 0, $-1$, 0,\dots, and so the Maclaurin series is
$$
  x-{x^3\over 3!}+{x^5\over 5!}-\cdots=
  \sum_{n=0}^\infty (-1)^n{x^{2n+1}\over (2n+1)!}.
$$
We should always determine the radius of convergence:
$$
  \lim_{n\to\infty} {|x|^{2n+3}\over (2n+3)!}{(2n+1)!\over |x|^{2n+1}}
  =\lim_{n\to\infty} {|x|^2\over (2n+3)(2n+2)}=0,
$$
so the series converges for every $x$. Since it turns out that this
series does indeed converge to $\sin x$ everywhere, we have a series
representation for $\sin x$ for every $x$. 
\expandafter\url\expandafter{\liveurl jsxgraph/taylor_series.html}%
Here is an interactive plot\endurl\ of the sine and some of its 
series approximations.
\end{example}

Sometimes the formula for the $n$th derivative of a function $f$ is
difficult to discover, but a combination of a known Maclaurin series
and some algebraic manipulation leads easily to the Maclaurin series
for $f$.

\begin{example} Find the Maclaurin series for $x\sin(-x)$.

To get from $\sin x$ to $x\sin(-x)$ we substitute $-x$ for $x$ and
then multiply by $x$. We can do the same thing to the series for $\sin
x$:
$$
  x\sum_{n=0}^\infty (-1)^n{(-x)^{2n+1}\over (2n+1)!}
  =x\sum_{n=0}^\infty (-1)^{n}(-1)^{2n+1}{x^{2n+1}\over (2n+1)!}
  =\sum_{n=0}^\infty (-1)^{n+1}{x^{2n+2}\over (2n+1)!}.
$$
\end{example}

As we have seen, a general power series can be centered at a point
other than zero, and the method that produces the Maclaurin series can
also produce such series.

\begin{example} Find a series centered at $-2$ for $1/(1-x)$.

If the series is $\ds\sum_{n=0}^\infty a_n(x+2)^n$ then looking at the
$k$th derivative:
$$k!(1-x)^{-k-1}=\sum_{n=k}^\infty {n!\over (n-k)!}a_n(x+2)^{n-k}$$
and substituting $x=-2$ we get
$\ds k!3^{-k-1}=k!a_k$ and $\ds a_k=3^{-k-1}=1/3^{k+1}$, so the series is
$$\sum_{n=0}^\infty {(x+2)^n\over 3^{n+1}}.$$
We've already seen this, on page~\xrefn{page:alt series for 1 over 1-x}.
\end{example}

Such a series is called the 
{\dfont Taylor series\index{Taylor series}\index{series!Taylor}\/} 
for the function,
and the general term has the form
$${f^{(n)}(a)\over n!}(x-a)^n.$$
A Maclaurin series is simply a Taylor series with $a=0$.

\begin{exercises}

For each function, find the Maclaurin series or Taylor series centered
at $a$, and the radius of convergence.

\begin{exercise} $\cos x$
\begin{answer} $\ds\sum_{n=0}^\infty (-1)^n x^{2n}/(2n)!$, $R=\infty$
\end{answer}\end{exercise}

\begin{exercise} $\ds e^x$
\begin{answer} $\ds\sum_{n=0}^\infty x^n/n!$, $R=\infty$
\end{answer}\end{exercise}

\begin{exercise} $1/x$, $a=5$
\begin{answer} $\ds\sum_{n=0}^\infty (-1)^n{(x-5)^n\over 5^{n+1}}$, $R=5$
\end{answer}\end{exercise}

\begin{exercise} $\ln x$, $a=1$
\begin{answer} $\ds\sum_{n=1}^\infty (-1)^{n-1}{(x-1)^n\over n}$, $R=1$
\end{answer}\end{exercise}

\begin{exercise} $\ln x$, $a=2$
\begin{answer} $\ds\ln(2)+\sum_{n=1}^\infty (-1)^{n-1}{(x-2)^n\over n 2^n}$, $R=2$
\end{answer}\end{exercise}

\begin{exercise} $\ds 1/x^2$, $a=1$
\begin{answer} $\ds\sum_{n=0}^\infty (-1)^n(n+1)(x-1)^n$, $R=1$
\end{answer}\end{exercise}

\begin{exercise} $\ds 1/\sqrt{1-x}$
\begin{answer} $\ds1+\sum_{n=1}^\infty {1\cdot3\cdot5\cdots(2n-1)\over
n!2^n} x^n=1+\sum_{n=1}^\infty {(2n-1)!\over 2^{2n-1}(n-1)!\,n!}x^n$, $R=1$
\end{answer}\end{exercise}

\begin{exercise} Find the first four terms of the Maclaurin series for $\tan
x$ (up to and including the $\ds x^3$ term).
\begin{answer} $\ds x+x^3/3$
\end{answer}\end{exercise}

\begin{exercise} Use a combination of Maclaurin series and algebraic
manipulation to find a series centered at zero for
$\ds x\cos (x^2)$.
\begin{answer} $\ds\sum_{n=0}^\infty (-1)^n x^{4n+1}/(2n)!$
\end{answer}\end{exercise}

\begin{exercise} Use a combination of Maclaurin series and algebraic
manipulation to find a series centered at zero for
$\ds xe^{-x}$.
\begin{answer} $\ds\sum_{n=0}^\infty (-1)^n x^{n+1}/n!$
\end{answer}\end{exercise}

\end{exercises}

