\section{Cylindrical and Spherical Coordinates}{}{}
\nobreak
We have seen that sometimes double integrals are simplified by doing
them in polar coordinates; not surprisingly, triple integrals are
sometimes simpler in cylindrical coordinates\index{cylindrical
  coordinates} or spherical coordinates\index{spherical coordinates}.
To set up integrals in polar coordinates, we had to understand the
shape and area of a typical small region into which the region of
integration was divided. We need to do the same thing here, for three
dimensional regions.

The cylindrical coordinate system is the simplest, since it is just
the polar coordinate system plus a $z$ coordinate. A typical small
unit of volume is the shape shown in figure~\xrefn{fig:cylindrical
  coordinates regions} ``fattened up'' in the $z$ direction, so its
volume is $r\Delta r\Delta \theta\Delta z$, or in the limit, 
$r\,dr\,d\theta\,dz$. 

\begin{example} Find the volume under $z=\sqrt{4-r^2}$ 
above the quarter circle inside $x^2+y^2=4$
in the first quadrant.

We could of course do this with a double integral, but we'll use a
triple integral:
$$\int_0^{\pi/2}\int_0^2\int_0^{\sqrt{4-r^2}} r\,dz\,dr\,d\theta=
\int_0^{\pi/2}\int_0^2 \sqrt{4-r^2}\; r\,dr\,d\theta=
{4\pi\over3}.$$
Compare this to example~\xrefn{example:integration in polar coordinates}.
\end{example}

\begin{example} An object occupies the space inside both the cylinder
$x^2+y^2=1$ and the sphere $x^2+y^2+z^2=4$, and has density $x^2$ at
$(x,y,z)$. Find the total mass.

We set this up in cylindrical coordinates, recalling that
$x=r\cos\theta$: 
$$\eqalign{
\int_0^{2\pi}\int_0^1\int_{-\sqrt{4-r^2}}^{\sqrt{4-r^2}}
r^3\cos^2(\theta)\,dz\,dr\,d\theta
&=\int_0^{2\pi}\int_0^1 2\sqrt{4-r^2}\;r^3\cos^2(\theta)\,dr\,d\theta \\
&=\int_0^{2\pi}
\left({128\over15}-{22\over5}\sqrt3\right)\cos^2(\theta)\,d\theta \\ 
&=\left({128\over15}-{22\over5}\sqrt3\right)\pi \\
}$$
\end{example}

Spherical coordinates are somewhat more difficult to understand. The
small volume we want will be defined by $\Delta\rho$, $\Delta\phi$,
and $\Delta\theta$, as pictured in figure~\xrefn{fig:spherical volume
  unit}.  To gain a better understanding, see the Java applet. The
small volume is nearly box shaped, with 4 flat sides and two sides
formed from bits of concentric spheres. When $\Delta\rho$, $\Delta\phi$,
and $\Delta\theta$ are all very small, the volume of this little
region will be nearly the volume we get by treating it as a box.
One dimension of the box is simply $\Delta\rho$, the change in distance
from the origin. The other two dimensions are the lengths of small
circular arcs, so they are $r\Delta\alpha$ for some suitable
$r$ and $\alpha$, just as in the polar coordinates case.

\figure
\vbox{\beginpicture
\normalgraphs
\ninepoint
\setcoordinatesystem units <1.5truecm,1.5truecm>
\setplotarea x from 0 to 2.1, y from -1.1 to 1.1
\put {\hbox{\epsfxsize9cm\epsfbox{spherical_volume_unit.eps}}} at 0 0
\endpicture}
\figrdef{fig:spherical volume unit}
\endfigure{A small unit of volume for spherical coordinates.
(\expandafter\url\expandafter{\liveurl spherical_volume_unit.html}%
AP\endurl)}

The easiest of these to understand is the arc corresponding to a
change in $\phi$, which is nearly identical to the derivation for
polar coordinates, as shown in the left graph in figure~\xrefn{fig:int
  spherical}. In that graph we are looking ``face on'' at the side of
the box we are interested in, so the small angle pictured is
precisely $\Delta\phi$, the vertical axis really is the $z$ axis, but
the horizontal axis is {\it not\/} a real axis---it is just some line
in the $x$-$y$ plane.
Because the other arc is governed by $\theta$, we need
to imagine looking straight down the $z$ axis, so that the apparent
angle we see is $\Delta\theta$. In this view, the axes really are the
$x$ and $y$ axes.
In this graph, the apparent distance from the
origin is not $\rho$ but $\rho\sin\phi$, as indicated in the left
graph. 

\figure
\vbox{\beginpicture
\normalgraphs
\ninepoint
\setcoordinatesystem units <1.5truecm,1.5truecm>
\setplotarea x from 0 to 3.5, y from 0 to 3
\axis left  /
\axis bottom  /
\circulararc 15 degrees from 3.17 1.48 center at 0 0
\circulararc 15 degrees from 2.266 1.057 center at 0 0
\setlinear
\plot 3.17 1.48 0 0 2.68 2.25 /
\betweenarrows {$\rho\sin\phi$} [t] <0pt,-3pt> from 0 0 to 3.17 0
\setdashes
\plot 3.17 1.48 3.17 0 /
\put {$z$} [b] <0pt,3pt> at 0 3
\put {$\Delta \rho$} [tl] <0pt,-3pt> at 2.72 1.27
\put {$\rho\Delta \phi$} [bl] <2pt,2pt> at 2.95 1.88
\put {\eightpoint$\Delta\phi$} at 1.01 0.645
\setsolid
\setcoordinatesystem units <1.5truecm,1.5truecm> point at -5 0
\setplotarea x from 0 to 3.5, y from 0 to 3
\axis left  /
\axis bottom  /
\circulararc 15 degrees from 2.87 1.34 center at 0 0
\circulararc 15 degrees from 2.05 0.955 center at 0 0
\setlinear
\plot 2.87 1.34 0 0 2.43 2.04 /
\put {$x$} [l] <3pt,0pt> at 3.5 0
\put {$y$} [b] <0pt,3pt> at 0 3
\put {$\rho\sin\phi\Delta \theta$} [bl] <2pt,2pt> at 2.67 1.7
\put {\eightpoint$\Delta\theta$} at 1.01 0.645
\endpicture}
\figrdef{fig:int spherical}
\endfigure{Setting up integration in spherical coordinates.}

The upshot is that the volume of the little box is approximately
$\Delta\rho(\rho\Delta\phi)(\rho\sin\phi\Delta\theta)
=\rho^2\sin\phi\Delta\rho\Delta\phi\Delta\theta$, or in the limit
$\rho^2\sin\phi\,d\rho\,d\phi\,d\theta$.

\begin{example} Suppose the temperature at $(x,y,z)$ is
$T=1/(1+x^2+y^2+z^2)$. Find the average temperature in the unit sphere
centered at the origin.

In two dimensions we add up the temperature at ``each'' point and
divide by the area; here we add up the temperatures and divide by the
volume, $(4/3)\pi$:
$${3\over4\pi}\int_{-1}^1\int_{-\sqrt{1-x^2}}^{\sqrt{1-x^2}}
\int_{-\sqrt{1-x^2-y^2}}^{\sqrt{1-x^2-y^2}}
{1\over1+x^2+y^2+z^2}\,dz\,dy\,dx
$$
This looks quite messy; since everything in the problem is closely
related to a sphere, we'll convert to spherical coordinates.
$${3\over4\pi}\int_0^{2\pi}\int_0^\pi
\int_0^1
{1\over1+\rho^2}\,\rho^2\sin\phi\,d\rho\,d\phi\,d\theta
={3\over4\pi}(4\pi -\pi^2)=3-{3\pi\over4}.
$$
\end{example}

\begin{exercises}

\begin{exercise} Evaluate $\ds\int_{0}^{1}\int_{0}^{x}\int_{0}^{\sqrt{x^2+y^2}}
{(x^2+y^2)^{3/2}\over x^2+y^2+z^2}\,dz\,dy\,dx$.
\begin{answer} $\pi/12$
\end{answer}\end{exercise}

% Albert
\begin{exercise} Evaluate $\ds\int_{-1}^{1}\int_{0}^{\sqrt{1-x^2}}
\int_{\sqrt{x^2+y^2}}^{\sqrt{2-x^2-y^2}}\sqrt{x^2+y^2+z^2}\,dz\,dy\,dx$.
\begin{answer} $\pi(1-\sqrt2/2)$
\end{answer}\end{exercise}
%/Albert

\begin{exercise} Evaluate $\ds\int\int\int x^2\,dV$
over the interior of the cylinder $x^2+y^2=1$ between $z=0$ and $z=5$.
\begin{answer} $5\pi/4$
\end{answer}\end{exercise}

\begin{exercise} Evaluate $\ds\int\int\int xy\,dV$
over the interior of the cylinder $x^2+y^2=1$ between $z=0$ and $z=5$.
\begin{answer} $0$
\end{answer}\end{exercise}

\begin{exercise} Evaluate $\ds\int\int\int z\,dV$
over the region above the $x$-$y$ plane, inside $x^2+y^2-2x=0$ and
under $x^2+y^2+z^2=4$.
\begin{answer} $5\pi/4$
\end{answer}\end{exercise}

\begin{exercise} Evaluate $\ds\int\int\int yz\,dV$
over the region in the first octant, inside $x^2+y^2-2x=0$ and 
under $x^2+y^2+z^2=4$.
\begin{answer} $4/5$
\end{answer}\end{exercise}

\begin{exercise} Evaluate $\ds\int\int\int x^2+y^2\,dV$
over the interior of $x^2+y^2+z^2=4$.
\begin{answer} $256\pi/15$
\end{answer}\end{exercise}

\begin{exercise} Evaluate $\ds\int\int\int \sqrt{x^2+y^2}\,dV$
over the interior of $x^2+y^2+z^2=4$.
\begin{answer} $4\pi^2$
\end{answer}\end{exercise}

\begin{exercise} Find the mass of a right circular cone of height $h$ and
base radius $a$ if the density is proportional to the distance from
the base.
\begin{answer} $\pi kh^2a^2/12$
\end{answer}\end{exercise}

\begin{exercise} Find the mass of a right circular cone of height $h$ and
base radius $a$ if the density is proportional to the distance from
its axis of symmetry.
\begin{answer} $\pi kha^3/6$
\end{answer}\end{exercise}

\begin{exercise} An object occupies the region inside the unit sphere at the
origin, and has density equal to the distance from the $x$-axis. Find
the mass.
\begin{answer} $\pi^2/4$
\end{answer}\end{exercise}

\begin{exercise} An object occupies the region inside the unit sphere at the
origin, and has density equal to the square of the distance from the
origin. Find the mass.
\begin{answer} $4\pi/5$
\end{answer}\end{exercise}

\begin{exercise} An object occupies the region between the unit sphere at the
origin and a sphere of radius 2 with center at the origin, and has
density equal to the distance from the origin. Find the mass.
\begin{answer} $15\pi$
\end{answer}\end{exercise}

%Albert
\begin{exercise} An object occupies the region in the first octant bounded by
the cones $\phi = \pi/4$ and $\phi = \arctan 2$, and the sphere $\rho
= \sqrt{6}$, and has density proportional to the distance from the
origin. Find the mass.
\begin{answer} $9k\pi(5\sqrt2-2\sqrt5)/20$
\end{answer}\end{exercise}
%/Albert

\end{exercises}

