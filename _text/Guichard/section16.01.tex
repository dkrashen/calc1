\section{Vector Fields}{}{}

This chapter is concerned with applying calculus in the context of 
{\dfont vector fields\index{vector fields}}. A two-dimensional vector
field is a function $f$ that maps each point $(x,y)$ in $\R^2$ to a
two-dimensional vector $\langle u,v\rangle$, and similarly a
three-dimensional vector field maps $(x,y,z)$ to $\langle
u,v,w\rangle$. Since a vector has no position, we typically indicate a
vector field in graphical form by placing the vector $f(x,y)$ with its
tail at $(x,y)$. Figure~\xrefn{fig:vector field} shows a
representation of the vector field 
$f(x,y)=\langle x/\sqrt{x^2+y^2+4},-y/\sqrt{x^2+y^2+4}\rangle$.
For such a graph to be readable, the vectors must be fairly short,
which is accomplished by using a different scale for the vectors than
for the axes. Such graphs are thus useful for understanding the sizes
of the vectors relative to each other but not their absolute size.


\figure
\vbox{\beginpicture
\normalgraphs
\ninepoint
\setcoordinatesystem units <3truecm,3truecm>
\setplotarea x from -1 to 1, y from 0 to 1
\put {\hbox{\epsfxsize8cm\epsfbox{vfield.eps}}} at 0 -0.15
\endpicture}
\figrdef{fig:vector field}
\endfigure{A vector field.}

Vector fields have many important applications, as they can be used to
represent many physical quantities: the vector at a point may
represent the strength of some force (gravity, electricity, magnetism) or
a velocity (wind speed or the velocity of some other fluid). 

We have already seen a particularly important kind of vector
field---the gradient. Given a function $f(x,y)$, recall that the
gradient is $\langle f_x(x,y),f_y(x,y)\rangle$, a vector that depends
on (is a function of) $x$ and $y$. We usually picture the gradient
vector with its tail at $(x,y)$, pointing in the direction of maximum
increase. Vector fields that are gradients have some particularly nice
properties, as we will see.
An important example is 
$${\bf F}=
\left
\langle {-x\over (x^2+y^2+z^2)^{3/2}},{-y\over (x^2+y^2+z^2)^{3/2}},{-z\over
  (x^2+y^2+z^2)^{3/2}}\right\rangle,$$
which points from the point $(x,y,z)$ toward the origin and has length
$${\sqrt{x^2+y^2+z^2}\over(x^2+y^2+z^2)^{3/2}}=
{1\over(\sqrt{x^2+y^2+z^2})^2},$$
which is the reciprocal of the square of the distance from $(x,y,z)$
to the origin---in other words, ${\bf F}$ is an ``inverse square
law''.
The vector $\bf F$ is a gradient:
$${\bf F} = \nabla {1\over\sqrt{x^2+y^2+z^2} },
\eqrdef{eq:inverse square field as gradient}
\eqno{(\xrefn{eq:inverse square field as gradient})}
$$
which turns out to be extremely useful.

\begin{exercises}

Sketch the vector fields; check your work with Sage's \vb|plot_vector_field|
function.

\begin{exercise} $\langle x,y\rangle$ 

\begin{exercise} $\langle -x, -y\rangle$ 

\begin{exercise} $\langle x,-y\rangle$ 

\begin{exercise} $\langle \sin x,\cos y\rangle$ 

\begin{exercise} $\langle y,1/x\rangle$ 

\begin{exercise} $\langle x+1,x+3\rangle$ 

\begin{exercise} Verify equation~\xrefn{eq:inverse square field as gradient}.

\end{exercises}

