\section{Divergence and Curl}{}{}

Divergence and curl are two measurements of vector fields that are
very useful in a variety of applications. Both are most easily
understood by thinking of the vector field as representing a flow of a
liquid or gas;
that is, each vector in the vector field should be interpreted as a
velocity vector. 
Roughly speaking, divergence
measures the tendency of 
the fluid to collect or disperse at a point, and curl measures the
tendency of the fluid to swirl around the point. Divergence is a
scalar, that is, a single number, while curl is itself a vector. The
magnitude of the curl measures how much the fluid is swirling, the
direction indicates the axis around which it tends to swirl. These
ideas are somewhat subtle in practice, and are beyond the scope of
this course. You can find additional information on the web, for
example at 
\url{http://mathinsight.org/curl_idea}%
\vb|http://mathinsight.org/curl_idea|\endurl\ 
and 
\url{http://mathinsight.org/divergence_idea}%
\vb|http://mathinsight.org/divergence_idea|\endurl\ 
and in
many books including {\it
Div, Grad, Curl, and All That: An Informal Text on Vector Calculus},
by H. M. Schey.

Recall that if $f$ is a function, the gradient of $f$
is given by 
$$\nabla f=\left\langle {\partial f\over\partial x},{\partial
  f\over\partial y},{\partial f\over\partial z}\right\rangle.$$
A useful mnemonic for this (and for the divergence and curl, as it
turns out) is to let
$$\nabla = \left\langle{\partial \over\partial x},{\partial
  \over\partial y},{\partial \over\partial z}\right\rangle,$$
that is, we pretend that $\nabla$ is a vector with rather odd looking
entries. Recalling that $\langle u,v,w\rangle a=\langle ua,va,wa\rangle$,
we can then think of the gradient as
$$\nabla f=\left\langle{\partial \over\partial x},{\partial
  \over\partial y},{\partial \over\partial z}\right\rangle f = 
\left\langle {\partial f\over\partial x},{\partial
  f\over\partial y},{\partial f\over\partial z}\right\rangle,$$
that is, we simply multiply the $f$ into the vector.

The divergence and curl can now be defined in terms of this same odd
vector $\nabla$ by using the cross product and dot product.
The divergence of a vector field ${\bf F}=\langle f,g,h\rangle$ is
$$\nabla \cdot {\bf F} =
\left\langle{\partial \over\partial x},{\partial
  \over\partial y},{\partial \over\partial z}\right\rangle\cdot
\langle f,g,h\rangle
= {\partial f\over\partial x}+{\partial
  g\over\partial y}+{\partial h\over\partial z}.$$
The curl of $\bf F$ is
$$\nabla\times{\bf F} = \left|\matrix{{\bf i}&{\bf j}&{\bf k} \\
{\partial \over\partial x}&{\partial
  \over\partial y}&{\partial \over\partial z} \\
f&g&h \\}\right| = 
\left\langle {\partial h\over\partial y}-{\partial g\over\partial z},
{\partial f\over\partial z}-{\partial h\over\partial x},
{\partial g\over\partial x}-{\partial f\over\partial y}\right\rangle.$$

Here are two simple but useful facts about divergence and curl.

\begin{theorem} \relax
\label{thm:div of curl is zero}
$\nabla\cdot(\nabla\times{\bf F})=0$.
\end{theorem}

In words, this says that the divergence of the curl is zero.

\begin{theorem} \relax
\label{thm:curl of gradient is zero}
$\nabla\times(\nabla f) = {\bf 0}$.
\end{theorem}

That is, the curl of a gradient is the zero vector. Recalling that
gradients are conservative vector fields, this says that the curl of a
conservative vector field is the zero vector. Under suitable
conditions, it is also true that if the curl of $\bf F$ is $\bf 0$
then $\bf F$ is conservative. (Note that this is exactly the same test
that we discussed on page~\xrefn{page:test for conservative vector field}.)

\begin{example} Let ${\bf F} = \langle e^z,1,xe^z\rangle$. Then 
$\nabla\times{\bf F} = \langle 0,e^z-e^z,0\rangle = {\bf 0}$.
Thus, $\bf F$ is conservative, and we can exhibit this directly by
finding the corresponding $f$.

Since $f_x=e^z$, $f=xe^z+g(y,z)$. Since $f_y=1$, it must be that
$g_y=1$, so $g(y,z)=y+h(z)$. Thus $f=xe^z+y+h(z)$ and
$$xe^z = f_z xe^z + 0 + h'(z),$$
so $h'(z)=0$, i.e., $h(z)=C$, and $f=xe^z+y+C$.
\end{example}

We can rewrite Green's Theorem using these new ideas; these rewritten
versions in turn are closer to some later theorems we will see.

Suppose we write a two dimensional vector field in the
form ${\bf F}=\langle P,Q,0\rangle$, where $P$ and $Q$ are functions
of $x$ and $y$. Then 
$$\nabla\times {\bf F} =
\left|\matrix{{\bf i}&{\bf j}&{\bf k} \\
{\partial \over\partial x}&{\partial
  \over\partial y}&{\partial \over\partial z} \\
P&Q&0 \\}\right|=
\langle 0,0,Q_x-P_y\rangle,$$
and so $(\nabla\times {\bf F})\cdot{\bf k}=\langle 0,0,Q_x-P_y\rangle\cdot
\langle 0,0,1\rangle = Q_x-P_y$. So Green's Theorem says
$$\int_{\partial D} {\bf F}\cdot d{\bf r}=
\int_{\partial D} P\,dx +Q\,dy = \dint{D} Q_x-P_y \,dA
=\dint{D}(\nabla\times {\bf F})\cdot{\bf k}\,dA.
\eqrdef{eq:greens theorem second form}
\eqno{(\xrefn{eq:greens theorem second form})}
$$
Roughly speaking, the right-most integral adds up the curl (tendency
to swirl) at each point in the region; the left-most integral adds up
the tangential components of the vector field around the entire
boundary. Green's Theorem says these are equal, or roughly, that the
sum of the ``microscopic'' swirls over the region is the same as the
``macroscopic'' swirl around the boundary.

Next, suppose that the boundary $\partial D$ has a vector form
${\bf r}(t)$, so that ${\bf r}'(t)$ is tangent to the boundary, and
${\bf T}={\bf r}'(t)/|{\bf r}'(t)|$ is the usual unit tangent vector.
Writing ${\bf r}=\langle x(t),y(t)\rangle$ we get
$${\bf T}={\langle x',y'\rangle\over|{\bf r}'(t)|}$$
and then
$${\bf N}={\langle y',-x'\rangle\over|{\bf r}'(t)|}$$
is a unit vector perpendicular to $\bf T$, that is, a unit normal to
the boundary. 
Now
$$\eqalign{
\int_{\partial D} {\bf F}\cdot{\bf N}\,ds&=
\int_{\partial D} \langle P,Q\rangle\cdot{\langle
  y',-x'\rangle\over|{\bf r}'(t)|} |{\bf r}'(t)|dt=
\int_{\partial D} Py'\,dt - Qx'\,dt \\
&=\int_{\partial D} P\,dy - Q\,dx
=\int_{\partial D} - Q\,dx+P\,dy. \\
}$$
So far, we've just rewritten the original integral using alternate
notation. The last integral looks just like the left side of Green's
Theorem (\xrefn{thm:greens theorem}) except that $P$ and $Q$ have
traded places and $Q$ has acquired a negative sign. Then applying
Green's Theorem we get 
$$
\int_{\partial D} - Q\,dx+P\,dy=\dint{D} P_x+Q_y\,dA=
\dint{D} \nabla\cdot{\bf F}\,dA.$$
Summarizing the long string of equalities, 
$$\int_{\partial D} {\bf F}\cdot{\bf N}\,ds=\dint{D} \nabla\cdot{\bf
  F}\,dA.
\eqrdef{eq:greens theorem third form}
\eqno{(\xrefn{eq:greens theorem third form})}
$$ 
Roughly speaking, the first integral adds up the flow across the
boundary of the region, from inside to out, and the second sums the
divergence (tendency to spread) at each point in the interior. The
theorem roughly says that the sum of the ``microscopic'' spreads is
the same as the total spread across the boundary and out of the region.

\begin{exercises}

\begin{exercise} Let ${\bf F}=\langle xy,-xy\rangle$ and 
let $D$ be given by $0\le x\le 1$, $0\le y\le 1$.
Compute $\ds\int_{\partial D} {\bf F}\cdot d{\bf r}$ and
$\ds\int_{\partial D} {\bf F}\cdot{\bf N}\,ds$.
\begin{answer} $-1$, $0$
\end{answer}\end{exercise}

\begin{exercise} Let ${\bf F}=\langle ax^2,by^2\rangle$ and 
let $D$ be given by $0\le x\le 1$, $0\le y\le 1$.
Compute $\ds\int_{\partial D} {\bf F}\cdot d{\bf r}$ and
$\ds\int_{\partial D} {\bf F}\cdot{\bf N}\,ds$.
\begin{answer} $0$, $a+b$
\end{answer}\end{exercise}

\begin{exercise} Let ${\bf F}=\langle ay^2,bx^2\rangle$ and 
let $D$ be given by $0\le x\le 1$, $0\le y\le x$.
Compute $\ds\int_{\partial D} {\bf F}\cdot d{\bf r}$ and
$\ds\int_{\partial D} {\bf F}\cdot{\bf N}\,ds$.
\begin{answer} $(2b-a)/3$, $0$
\end{answer}\end{exercise}

\begin{exercise} Let ${\bf F}=\langle \sin x\cos y,\cos x\sin y\rangle$ and 
let $D$ be given by $0\le x\le \pi/2$, $0\le y\le x$.
Compute $\ds\int_{\partial D} {\bf F}\cdot d{\bf r}$ and
$\ds\int_{\partial D} {\bf F}\cdot{\bf N}\,ds$.
\begin{answer} $0$, $1$
\end{answer}\end{exercise}

\begin{exercise} Let ${\bf F}=\langle y,-x\rangle$ and 
let $D$ be given by $x^2+y^2\le 1$.
Compute $\ds\int_{\partial D} {\bf F}\cdot d{\bf r}$ and
$\ds\int_{\partial D} {\bf F}\cdot{\bf N}\,ds$.
\begin{answer} $-2\pi$, $0$
\end{answer}\end{exercise}

\begin{exercise} Let ${\bf F}=\langle x,y\rangle$ and 
let $D$ be given by $x^2+y^2\le 1$.
Compute $\ds\int_{\partial D} {\bf F}\cdot d{\bf r}$ and
$\ds\int_{\partial D} {\bf F}\cdot{\bf N}\,ds$.
\begin{answer} $0$, $2\pi$
\end{answer}\end{exercise}

\begin{exercise} Prove theorem~\xrefn{thm:div of curl is zero}.

\begin{exercise} Prove theorem~\xrefn{thm:curl of gradient is zero}.

\begin{exercise} If $\nabla \cdot F=0$, $F$ is said to be {\dfont
incompressible\index{incompressible}}.  Show that any vector field
of the form $F(x,y,z) = \langle f(y,z),g(x,z),h(x,y\rangle$ is
incompressible.  Give a non-trivial example.

\end{exercises}


