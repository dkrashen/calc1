\section{Surface Integrals}{}{}

In the integral for surface area,
$$\int_a^b\int_c^d |{\bf r}_u\times{\bf r}_v|\,du\,dv,$$
the integrand $|{\bf r}_u\times{\bf r}_v|\,du\,dv$
is the area of a tiny parallelogram, that is, a very small surface
area, so it is reasonable to abbreviate it $dS$; then a shortened
version of the integral is
$$\dint{D} 1\cdot dS.$$
We have already seen that if $D$ is a region in the plane, the area of
$D$ may be computed with 
$$\dint{D} 1\cdot dA,$$
so this is really quite familiar, but the $dS$ hides a little more
detail than does $dA$.

Just as we can integrate functions $f(x,y)$ over regions in the plane,
using
$$\dint{D} f(x,y)\, dA,$$
so we can compute integrals over surfaces in space, using
$$\dint{D} f(x,y,z)\, dS.$$
In practice this means that we have a vector function
${\bf r}(u,v)=\langle x(u,v),y(u,v),z(u,v)\rangle$ for the surface,
and the integral we compute is
$$\int_a^b\int_c^d f(x(u,v),y(u,v),z(u,v))|{\bf r}_u\times{\bf
  r}_v|\,du\,dv.$$ 
That is, we express everything in terms of $u$ and $v$, and then we
can do an ordinary double integral.

\begin{example} Suppose a thin object occupies the upper hemisphere of 
$x^2+y^2+z^2=1$ and has density $\sigma(x,y,z)=z$. Find the mass 
and center of mass of the
object. (Note that the object is just a thin shell; it does not occupy
the interior of the hemisphere.)

We write the hemisphere as ${\bf r}(\phi,\theta)=
\langle \cos\theta\sin\phi, \sin\theta\sin\phi, \cos\phi\rangle$,
$0\le\phi\le \pi/2$ and $0\le\theta\le 2\pi$. So
${\bf r}_\theta = \langle -\sin\theta\sin\phi, \cos\theta\sin\phi, 0\rangle$
and 
${\bf r}_\phi =\langle \cos\theta\cos\phi, \sin\theta\cos\phi, -\sin\phi\rangle$.
Then
$${\bf r}_\theta\times{\bf r}_\phi =
\langle -\cos\theta\sin^2\phi,-\sin\theta\sin^2\phi,-\cos\phi\sin\phi\rangle$$
and
$$ |{\bf r}_\theta\times{\bf r}_\phi| = |\sin\phi| = \sin\phi,$$
since we are interested only in $0\le\phi\le \pi/2$.
Finally, the density is $z=\cos\phi$ and the integral for mass is
$$\int_0^{2\pi}\int_0^{\pi/2} \cos\phi\sin\phi\,d\phi\,d\theta=\pi.$$

By symmetry, the center of mass is clearly on the $z$-axis, so we only
need to find the $z$-coordinate of the center of mass. The moment
around the $x$-$y$ plane is
$$\int_0^{2\pi}\int_0^{\pi/2} z\cos\phi\sin\phi\,d\phi\,d\theta
=\int_0^{2\pi}\int_0^{\pi/2} \cos^2\phi\sin\phi\,d\phi\,d\theta
={2\pi\over 3},$$
so the center of mass is at $(0,0,2/3)$.
\end{example}

Now suppose that ${\bf F}$ is a vector field; imagine that it
represents the velocity of some fluid at each point in space. We would
like to measure how much fluid is passing through a surface $D$, the
{\dfont flux\index{flux}\/} across $D$. As usual, we imagine computing
the flux across a very small section of the surface, with area $dS$,
and then adding up all such small fluxes over $D$ with an
integral. Suppose that vector $\bf N$ is a unit normal to the surface
at a point; ${\bf F}\cdot{\bf N}$ is the scalar projection of $\bf F$
onto the direction of $\bf N$, so it measures how fast the fluid is
moving across the surface. In one unit of time the fluid moving across
the surface will fill a volume of ${\bf F}\cdot{\bf N}\,dS$, which is
therefore the rate at which the fluid is moving across a small patch
of the surface. Thus, the total flux across $D$ is
$$\dint{D} {\bf F}\cdot{\bf N}\,dS=\dint{D} {\bf F}\cdot\,d{\bf S},$$
defining $d{\bf S}={\bf N}\,dS$.
As usual, certain conditions must be met for this to work out; chief
among them is the nature of the surface. As we integrate over the
surface, we must choose the normal vectors $\bf N$ in such a way that
they point ``the same way'' through the surface. For example, if the
surface is roughly horizontal in orientation, we might want to measure
the flux in the ``upwards'' direction, or if the surface is closed,
like a sphere, we might want to measure the flux ``outwards'' across
the surface. In the first case we would choose $\bf N$ to have
positive $z$ component, in the second we would make sure that $\bf N$
points away from the origin. Unfortunately, there are surfaces that
are not {\dfont orientable\index{orientable surface}\/}: they have
only one side, so that it is not possible to choose the normal vectors
to point in the ``same way'' through the surface. The most famous such
surface is the M\"obius strip shown in figure~\xrefn{fig:moebius}. It
is quite easy to make such a strip with a piece of paper and some
tape. If you have never done this, it is quite instructive; in
particular, you should draw a line down the center of the strip until
you return to your starting point. No matter how unit normal vectors
are assigned to the points of the M\"obius strip, there will be normal
vectors very close to each other pointing in opposite directions.

\figure
\vbox{\beginpicture
\normalgraphs
\ninepoint
\setcoordinatesystem units <2.5truecm,2.5truecm>
\setplotarea x from -1 to 1, y from 0 to 1
\put {\hbox{\epsfxsize7cm\epsfbox{moebius.eps}}} at 0 0
\endpicture}
\figrdef{fig:moebius}
\endfigure{A M\"obius strip.
(\expandafter\url\expandafter{\liveurl moebius.html}%
AP\endurl)}

Assuming that the quantities involved are well behaved, however, the
flux of the vector field across the surface ${\bf r}(u,v)$ is
$$\dint{D} {\bf F}\cdot{\bf N}\,dS
=\dint{D}{\bf F}\cdot 
 {{\bf r}_u\times{\bf r}_v\over|{\bf r}_u\times{\bf r}_v|}
 |{\bf r}_u\times{\bf r}_v|\,dA
=\dint{D}{\bf F}\cdot ({\bf r}_u\times{\bf r}_v)\,dA.$$
In practice, we may have to use ${\bf r}_v\times{\bf r}_u$
or even something a bit more complicated to make sure that the normal
vector points in the desired direction.

\begin{example} Compute the flux of ${\bf F}=\langle x,y,z^4\rangle$ across the
cone $z=\sqrt{x^2+y^2}$, $0\le z\le 1$, in the downward direction.

We write the cone as a vector function: ${\bf r}=\langle v\cos u, v\sin u,
v\rangle$, $0\le u\le 2\pi$ and $0\le v\le 1$.
Then ${\bf r}_u=\langle -v\sin u, v\cos u,0\rangle$ and 
${\bf r}_v=\langle \cos u, \sin u, 1\rangle$ and
${\bf r}_u\times{\bf r}_v=\langle v\cos u,v\sin u,-v\rangle$.
The third coordinate $-v$ is negative, which is exactly what we
desire, that is, the normal vector points down through the
surface. Then 
$$\eqalign{
\int_0^{2\pi}\int_0^1 \langle x,y,z^4\rangle\cdot\langle v\cos
u,v\sin u,-v\rangle \,dv\,du
&=\int_0^{2\pi}\int_0^1 xv\cos u+yv\sin u-z^4v\,dv\,du \\
&=\int_0^{2\pi}\int_0^1 v^2\cos^2 u+ v^2\sin^2 u-v^5\,dv\,du \\
&=\int_0^{2\pi}\int_0^1 v^2-v^5\,dv\,du={\pi\over3}. \\
}$$
\end{example}

\begin{exercises}

\begin{exercise} Find the center of mass of an object that occupies the upper
hemisphere of $x^2+y^2+z^2=1$ and has density $x^2+y^2$.
\begin{answer} $(0,0,3/8)$
\end{answer}\end{exercise}

\begin{exercise} Find the center of mass of an object that occupies the
surface $z=xy$, $0\le x\le1$, $0\le y\le 1$ and has density $\sqrt{1+x^2+y^2}$.
\begin{answer} $(11/20,11/20,3/10)$
\end{answer}\end{exercise}

% Albert
\begin{exercise} Find the center of mass of an object that occupies the
surface $\ds z=\sqrt{x^2+y^2}$, $1\le z\le4$ and has density $x^2z$.
\begin{answer} $(0,0,1364/425)$
\end{answer}\end{exercise}
%/Albert

\begin{exercise} Find the centroid of the surface of a right circular cone of
height $h$ and base radius $r$, not including the base.
\begin{answer} on center axis, $h/3$ above the base
\end{answer}\end{exercise}

\begin{exercise} Evaluate $\ds \dint{D} \langle 2,-3,4\rangle\cdot {\bf
  N}\,dS$, where $D$ is given by $z=x^2+y^2$, $-1\le x\le 1$, $-1\le
y\le 1$, oriented up.
\begin{answer} $16$
\end{answer}\end{exercise}

\begin{exercise} Evaluate $\ds \dint{D} \langle x,y,3\rangle\cdot {\bf
  N}\,dS$, where $D$ is given by $z=3x-5y$, $1\le x\le 2$, $0\le
y\le 2$, oriented up.
\begin{answer} $7$
\end{answer}\end{exercise}

\begin{exercise} Evaluate $\ds \dint{D} \langle x,y,-2\rangle\cdot {\bf
  N}\,dS$, where $D$ is given by $z=1-x^2-y^2$, $x^2+y^2\le1$,
oriented up.
\begin{answer} $-\pi$
\end{answer}\end{exercise}

\begin{exercise} Evaluate $\ds \dint{D} \langle xy, yz,zx\rangle\cdot {\bf
  N}\,dS$, where $D$ is given by $z=x+y^2+2$, $0\le x\le 1$, $x\le
y\le 1$, oriented up.
\begin{answer} $-137/120$
\end{answer}\end{exercise}

\begin{exercise} Evaluate $\ds \dint{D} \langle e^x, e^y,z\rangle\cdot {\bf
  N}\,dS$, where $D$ is given by $z=xy$, $0\le x\le 1$, $-x\le
y\le x$, oriented up.
\begin{answer} $-2/e$
\end{answer}\end{exercise}

\begin{exercise} Evaluate $\ds \dint{D} \langle xz,yz,z\rangle\cdot {\bf
N}\,dS$, where $D$ is given by $z=a^2-x^2-y^2$, $x^2+y^2\le b^2$, 
oriented up.
\begin{answer} $\pi b^2(-4b^4-3b^2+6a^2b^2+6a^2)/6$
\end{answer}\end{exercise}

\begin{exercise} A fluid has density 870 kg/m$^3$ and flows with velocity ${\bf v} =
 \langle z,y^2,x^2\rangle$, where distances are in meters and the
 components of ${\bf v}$ are in meters per second.  Find the rate of flow
 outward through the portion of the cylinder $x^2+y^2 = 4$, $0\leq
 z\leq 1$ for which $y>0$.
\begin{answer} $4/3$ kg/s
\end{answer}\end{exercise}

\begin{exercise} Gauss's Law says that the net charge, $Q$,
enclosed by a closed surface, $S$, is 
$$Q=\epsilon_0 \dint{} {\bf E}\cdot {\bf N}\,dS$$ 
where ${\bf E}$ is an electric field and $\epsilon_0$ (the
permittivity of free space) is a known constant; {\bf N} is oriented
outward. 
Use Gauss's Law to find the charge contained in the cube with vertices
$(\pm 1, \pm 1, \pm 1)$ if the electric field is 
${\bf E} = \langle x,y,z\rangle$.
\begin{answer} $24\epsilon_0$
\end{answer}\end{exercise}

\end{exercises}

