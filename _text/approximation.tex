\chapter{Approximation}

\section{Linear Approximation and Differentials}



Given a function, a \textit{linear approximation} is a fancy phrase
for something you already know.

\begin{definition}\index{linear approximation}
If $f(x)$ is a differentiable function at $x=a$, then a \textbf{linear
  approximation} for $f(x)$ at $x=a$ is given by
\[
\l(x) = f'(a)(x-a) +f(a).
\]
\end{definition}

A linear approximation of $f(x)$ is a good approximation of $f(x)$ as
long as $x$ is ``not too far'' from $a$.  As we see from
Figure~\ref{figure:informal-tangent}, if one can ``zoom in'' on $f(x)$
sufficiently, then $f(x)$ and the linear approximation are nearly
indistinguishable. Linear approximations allow us to make approximate
``difficult'' computations.

\begin{example}
Use a linear approximation of $f(x) =\sqrt[3]{x}$ at $x=64$ to
approximate $\sqrt[3]{50}$.
\end{example}



\begin{marginfigure}
\begin{tikzpicture}
	\begin{axis}[
            xmin=1,xmax=100,ymin=0,ymax=5,
            axis lines=center,
            xlabel=$x$, ylabel=$y$,
            every axis y label/.style={at=(current axis.above origin),anchor=south},
            every axis x label/.style={at=(current axis.right of origin),anchor=west},
          ]        
          \addplot [very thick, penColor, samples=150,smooth,domain=(0:100)] {x^(1/3))};
          \addplot [very thick, penColor2, domain=(0:100)] {x/48+8/3};
          \addplot [textColor,dashed] plot coordinates {(64,0) (64,4)};
          \addplot [textColor,dashed] plot coordinates {(0,4) (64,4)};
          \node at (axis cs:20,2.3) [penColor] {$f(x)$};
          \node at (axis cs:20,3.3) [penColor2] {$\l(x)$};
          \addplot[color=penColor3,fill=penColor3,only marks,mark=*] coordinates{(64,4)};  %% closed hole            
        \end{axis}
\end{tikzpicture}
\caption{A linear approximation of $f(x) = \sqrt[3]{x}$ at $x=64$.}
\label{figure:la sqrt3x}
\end{marginfigure}


\begin{solution}
To start, write
\[
\ddx f(x) = \ddx x^{1/3} = \frac{1}{3x^{2/3}}.
\]
so our linear approximation is
\begin{align*}
\l(x) &= \frac{1}{3\cdot 64^{2/3}} (x-64) + 4 \\
&= \frac{1}{48} (x-64) + 4\\
&= \frac{x}{48} +\frac{8}{3}.
\end{align*}
Now we evaluate $\l(50) \approx 3.71$ and compare it to
$\sqrt[3]{50}\approx 3.68$, see Figure~\ref{figure:la sqrt3x}. From
this we see that the linear approximation, while perhaps inexact, is
computationally \textbf{easier} than computing the cube root.
\end{solution}

With modern calculators and computing software it may not appear
necessary to use linear approximations. But in fact they are quite
useful. In cases requiring an explicit numerical approximation, they
allow us to get a quick rough estimate which can be used as a
``reality check'' on a more complex calculation. In some complex
calculations involving functions, the linear approximation makes an
otherwise intractable calculation possible, without serious loss of
accuracy.




\begin{example}%\label{exam:linear approximation of sine}
Use a linear approximation of $f(x) =\sin(x)$ at $x=0$ to approximate
$\sin(0.3)$.
\end{example}

\begin{marginfigure}
\begin{tikzpicture}
	\begin{axis}[
            xmin=-1.6,xmax=1.6,ymin=-1.5,ymax=1.5,
            axis lines=center,
            xtick={-1.57, 0, 1.57},
            xticklabels={$-\pi/2$, $0$, $\pi/2$},
            ytick={-1,1},
            %ticks=none,
            %width=3in,
            %height=2in,
            unit vector ratio*=1 1 1,
            xlabel=$x$, ylabel=$y$,
            every axis y label/.style={at=(current axis.above origin),anchor=south},
            every axis x label/.style={at=(current axis.right of origin),anchor=west},
          ]        
          \addplot [very thick, penColor, samples=100,smooth, domain=(-1.6:1.6)] {sin(deg(x))};
          \addplot [very thick, penColor2, samples=100,smooth] {x};
          \node at (axis cs:1,.6) [penColor] {$f(x)$};
          \node at (axis cs:-1,-1.2) [penColor2] {$\l(x)$};
          \addplot[color=penColor3,fill=penColor3,only marks,mark=*] coordinates{(0,0)};  %% closed hole          
        \end{axis}
\end{tikzpicture}
\caption{A linear approximation of $f(x) = \sin(x)$ at $x=0$.}
\label{figure:la sin}
\end{marginfigure}

\begin{solution}
To start, write
\[
\ddx f(x) = \cos(x),
\]
so our linear approximation is
\begin{align*}
\l(x) &= \cos(0)\cdot(x-0) + 0\\
&= x.
\end{align*}
Hence a linear approximation for $\sin(x)$ at $x=0$ is $\l(x) = x$,
and so $\l(0.3) = 0.3$.  Comparing this to $\sin(.3) \approx 0.295$. As
we see the approximation is quite good. For this reason, it is common
to approximate $\sin(x)$ with its linear approximation $\l(x) = x$
when $x$ is near zero, see Figure~\ref{figure:la sin}.
\end{solution}


\subsection*{Differentials}

The notion of a \textit{differential} goes back to the origins of
calculus, though our modern conceptualization of a differential is
somewhat different than how they were initially understood.

\begin{definition}\index{differential}
Let $f(x)$ be a differentiable function. We define a new
  independent variable $dx$, and a new dependent variable
\[
dy=f'(x)\cdot dx.
\] 
The variables $dx$ and $dy$ are called \textbf{differentials}, see
Figure~\ref{figure:differentials}.
\end{definition}
\begin{marginfigure}[0in]
\begin{tikzpicture}
	\begin{axis}[
            xmin=1, xmax=2, range=0:6,ymax=6,ymin=0,
            axis lines =left, xlabel=$x$, ylabel=$y$,
            every axis y label/.style={at=(current axis.above origin),anchor=south},
            every axis x label/.style={at=(current axis.right of origin),anchor=west},
            ticks=none,
            axis on top,
          ]         
	  \addplot [penColor2,very thick] plot coordinates {(1.4,10/6) (1.7,10/6)};
          \addplot [penColor2,very thick] plot coordinates {(1.7,10/6) (1.7,10/6 +.3/.36)};
          \addplot [penColor,dashed] plot coordinates {(1.4,10/6) (1.7,10/6 +.3/.36)};
          \addplot [very thick,penColor, smooth,samples=100,domain=(0:1.833)] {-1/(x-2)};
          \addplot[color=penColor3,fill=penColor3,only marks,mark=*] coordinates{(1.4,10/6)};  %% closed hole            
          \node at (axis cs:1.55,1.67) [below, penColor2] {$dx$};
          \node at (axis cs:1.7,2.08) [right, penColor2] {$dy$};
        \end{axis}
\end{tikzpicture}
\caption{While $dy$ and $dx$ are both variables, $dy$ depends on $dx$,
  and approximates how much a function grows after a change of size
  $dx$ from a given point.}
\label{figure:differentials}
\end{marginfigure}

Note, it is now the case (by definition!) that 
\[
\frac{dy}{dx} = f'(x).
\]

Essentially, differentials allow us to solve the problems presented in
the previous examples from a slightly different point of view. Recall,
when $h$ is near but not equal zero,
\[
f'(x) \approx \frac{f(x+h)-f(x)}{h}
\]
hence, 
\[
f'(x)h \approx f(x+h)-f(x)
\]
since $h$ is simply a variable, and $dx$ is simply a variable, we can replace $h$ with $dx$ to write
\begin{align*}
f'(x)\cdot dx &\approx f(x+dx)-f(x)\\
dy &\approx f(x+dx)-f(x).
\end{align*}
From this we see that 
\[
f(x+dx)\approx dy + f(x).
\]
While this is something of a ``sleight of hand'' with variables, there
are contexts where the language of differentials is common. We will
repeat our previous examples using differentials.

\begin{example}
Use differentials to approximate $\sqrt[3]{50}$.
\end{example}

\begin{marginfigure}
\begin{tikzpicture}
	\begin{axis}[
            xmin=1,xmax=100,ymin=2,ymax=5,
            axis lines=center,
            xlabel=$x$, ylabel=$y$,
            every axis y label/.style={at=(current axis.above origin),anchor=south},
            every axis x label/.style={at=(current axis.right of origin),anchor=west},
          ]        
          \addplot [very thick, penColor, samples=150,smooth,domain=(0:100)] {x^(1/3))};
          %\addplot [very thick, penColor2, domain=(50:64)] {x/48+8/3};
          %\addplot [textColor,dashed] plot coordinates {(64,0) (64,4)};
          \node at (axis cs:20,2.3) [penColor] {$f(x)$};
          \addplot [penColor2,very thick] plot coordinates {(64,4) (64,3.71)};
          \addplot [penColor2,very thick] plot coordinates {(50,3.71) (64,3.71)};
          \node [below] at (axis cs:57,3.7) [penColor2] {$dx$};
          \node [right] at (axis cs:64,3.8) [penColor2] {$dy$};
          \addplot[color=penColor3,fill=penColor3,only marks,mark=*] coordinates{(64,4)};  %% closed hole            
        \end{axis}
\end{tikzpicture}
\caption{A plot of $f(x) = \sqrt[3]{x}$  along with the differentials $dx$ and
  $dy$.}
\label{figure:diff sqrt3x}
\end{marginfigure}


\begin{solution}
Since $4^3 = 64$ is a perfect cube near $50$, we will set $dx=-14$. In this case
\[
\frac{dy}{dx} = f'(x)  = \frac{1}{3x^{2/3}}
\]
hence 
\begin{align*}
dy &= \frac{1}{3x^{2/3}} \cdot dx\\
&= \frac{1}{3\cdot64^{2/3}} \cdot(-14)\\
&= \frac{1}{3\cdot64^{2/3}} \cdot(-14)\\
&= \frac{-7}{24}
\end{align*}
Now $f(50) \approx f(64) + \frac{-7}{24} \approx 3.71$.
\end{solution}


\begin{example}
Use differentials to approximate $\sin(0.3)$.
\end{example}

\begin{marginfigure}
\begin{tikzpicture}
	\begin{axis}[
            xmin=-1.6,xmax=1.6,ymin=-1.5,ymax=1.5,
            axis lines=center,
            xtick={-1.57, 0, 1.57},
            xticklabels={$-\pi/2$, $0$, $\pi/2$},
            ytick={-1,1},
            %ticks=none,
            %width=3in,
            %height=2in,
            unit vector ratio*=1 1 1,
            xlabel=$x$, ylabel=$y$,
            every axis y label/.style={at=(current axis.above origin),anchor=south},
            every axis x label/.style={at=(current axis.right of origin),anchor=west},
          ]        
          \addplot [very thick, penColor, samples=100,smooth, domain=(-1.6:1.6)] {sin(deg(x))};
          %\addplot [very thick, penColor2, domain={0:0.3}] {x};
          \addplot [penColor2,very thick] plot coordinates {(0,0) (.3,0)};
          \addplot [penColor2,very thick] plot coordinates {(.3,0) (.3,.3)};
          \node at (axis cs:1,.6) [penColor] {$f(x)$};
          \node [below] at (axis cs:.15,.0) [penColor2] {$dx$};
          \node [right] at (axis cs:.3,.15) [penColor2] {$dy$};
          \addplot[color=penColor3,fill=penColor3,only marks,mark=*] coordinates{(0,0)};  %% closed hole          
        \end{axis}
\end{tikzpicture}
\caption{A plot of $f(x) = \sin(x)$ along with the differentials $dx$ and
  $dy$.}
\label{figure:diff sin}
\end{marginfigure}


\begin{solution}
Since $\sin(0) = 0$, we will set $dx=0.3$. In this case
\[
\frac{dy}{dx} = f'(x)  = \cos(x)
\]
hence 
\begin{align*}
dy &= \cos(0) \cdot dx\\
&= 1 \cdot (0.3)\\
&= 0.3
\end{align*}
Now $f(.3) \approx f(0) + 0.3 \approx 0.3$.
\end{solution}

The upshot is that linear approximations and differentials are simply
two slightly different ways of doing the exact same thing.





\begin{exercises}

\begin{exercise}
Use a linear approximation of $f(x) =\sin(x/2)$ at $x=0$ to approximate
$f(0.1)$.
\begin{answer}
$\sin(0.1/2)\approx 0.05$
\end{answer}
\end{exercise}

\begin{exercise}
Use a linear approximation of $f(x) =\sqrt[3]{x}$ at $x=8$ to approximate
$f(10)$.
\begin{answer}
$\sqrt[3]{10}\approx 2.17$
\end{answer}
\end{exercise}

\begin{exercise}
Use a linear approximation of $f(x) =\sqrt[5]{x}$ at $x=243$ to approximate
$f(250)$.
\begin{answer}
$\sqrt[5]{250}\approx 3.017$
\end{answer}
\end{exercise}


\begin{exercise}
Use a linear approximation of $f(x) =\ln(x)$ at $x=1$ to approximate
$f(1.5)$.
\begin{answer}
$\ln(1.5)\approx 0.5$
\end{answer}
\end{exercise}

\begin{exercise}
Use a linear approximation of $f(x) =\ln(\sqrt{x})$ at $x=1$ to approximate
$f(1.5)$.
\begin{answer}
$\ln(\sqrt{1.5})\approx 0.25$
\end{answer}
\end{exercise}


\begin{exercise} 
Let $f(x) = \sin(x/2)$. If $x=1$ and $dx=1/2$, what is $dy$?
\begin{answer} $dy=0.22$
\end{answer}\end{exercise}

\begin{exercise} 
Let $f(x) = \sqrt{x}$. If $x=1$ and $dx =1/10$, what is $dy$?
\begin{answer} $dy=0.05$
\end{answer}\end{exercise}

\begin{exercise} 
Let $f(x) = \ln(x)$. If $x=1$ and $dx =1/10$, what is $dy$?
\begin{answer} $dy=0.1$
\end{answer}\end{exercise}


\begin{exercise} 
Let $f(x) = \sin (2x)$. If $x=\pi$ and $dx=\pi/100$, what is $dy$?
\begin{answer} $dy=\pi/50$
\end{answer}\end{exercise}

\begin{exercise} Use differentials to estimate the amount of paint needed to
 apply a coat of paint 0.02 cm thick to a sphere with diameter $40$
 meters. Hint: Recall that the volume of a sphere of radius $r$ is $V
 =(4/3)\pi r^3$. Note that you are given that $dr=0.02$ cm.
\begin{answer} $dV=8\pi/25 \text{m}^3$
\end{answer}\end{exercise}

\end{exercises}













\section{Newton's Method}

Suppose you have a function $f(x)$, and you want to solve $f(x)=0$.
Solving equations symbolically is difficult. However, Newton's method
gives us a procedure, for finding a solution to many equations to as
many decimal places as you want.  

\marginnote[.5in]{The point
\[
a_{n+1} = a_n - \frac{f(a_n)}{f'(a_n)}
\]
is the solution to the equation $\l_n(x) = 0$, where $\l_n(x)$ is the
linear approximation of $f(x)$ at $x=a_n$.}

\begin{newtonsMethod}
Let $f(x)$ be a differentiable function and let $a_0$ be a
guess for a solution to the equation
\[
f(x) = 0.
\]
We can produce a sequence of points $x=a_0, a_1, a_2, a_3, \dots$ via
the recursive formula
\[
a_{n+1} = a_n -\frac{f(a_n)}{f'(a_n)}
\]
that (hopefully!) are successively better approximations of a solution
to the equation $f(x) = 0$.
\end{newtonsMethod}
Let's see if we can explain the logic behind this method. Consider the
following cubic function
\[
f(x) = x^3 - 4 x^2 - 5 x - 7.
\]
While there is a ``cubic formula'' for finding roots, it can be
difficult to implement. Since it is clear that $f(10)$ is positive,
and $f(0)$ is negative, by the Intermediate Value
Theorem~\ref{theorem:IVT}, there is a solution to the equation $f(x) =
0$ in the interval $[0,10]$. Let's compute $f'(x) = 3x^2 -8x-5$ and
guess that $a_0=7$ is a solution. We can easily see that
\[
f(a_0) = f(7) = 105\qquad\text{and}\qquad f'(a_0) = f'(7) = 86.
\]
This might seem pretty bad, but if we look at the linear approximation
of $f(x)$ at $x=7$, we find
\begin{marginfigure}[-1in]
\begin{tikzpicture}
	\begin{axis}[
            xmin=4, xmax=7.5,ymin=-40,ymax=125,domain=(4:8),
            axis lines =center, xlabel=$x$, ylabel=$y$,
            every axis y label/.style={at=(current axis.above origin),anchor=south},
            every axis x label/.style={at=(current axis.right of origin),anchor=west},
            xtick={7}, ytickmin=1,ytickmax=0,
            xticklabels={$a_0$},
            axis on top,
          ]         
	  \addplot [penColor2,very thick] {86*(x-7) + 105};
          \addplot [textColor,dashed] plot coordinates {(7,0) (7,105)};
          \addplot [very thick,penColor, smooth,samples=100] {x^3 - 4*x^2 - 5*x - 7};
          \addplot[color=penColor3,fill=penColor3,only marks,mark=*] coordinates{(7,105)};  %% closed hole          
        \end{axis}
\end{tikzpicture}
\caption{Here we see our first guess, along with the linear
  approximation at that point.}
\label{figure:newtonCubic1}
\end{marginfigure}
\[
\l_0(x) = 86(x-7) + 105 \qquad\text{which is the same as}\qquad \l_0(x) = f'(a_0)(x-a_0) + f(a_0).
\]
Now $\l_0(a_1) = 0$ when
\[
a_1 = 7 - \frac{105}{86} \qquad\text{which is the same as}\qquad a_1 = a_0
-\frac{f(a_0)}{f'(a_0)}.
\]
To remind you what is going on geometrically see
Figure~\ref{figure:newtonCubic1}. Now we repeat the procedure letting
$a_1$ be our new guess. Now
\[
f(a_1) \approx 23.5.
\]
We see our new guess is better than our first. If we look at the
linear approximation of $f(x)$ at $x=a_1$, we find
\[
\l_1(x) = f'(a_1)(x-a_1) + f(a_1).
\]
Now $\l_1(a_2) = 0$ when
\[
a_2 = a_1 - \frac{f(a_1)}{f'(a_1)}.
\]

\begin{marginfigure}[-2in]
\begin{tikzpicture}
	\begin{axis}[
            xmin=4, xmax=7.5,ymin=-40,ymax=125,domain=(4:8),
            axis lines =center, xlabel=$x$, ylabel=$y$,
            every axis y label/.style={at=(current axis.above origin),anchor=south},
            every axis x label/.style={at=(current axis.right of origin),anchor=west},
            xtick={5.78,7}, ytickmin=1,ytickmax=0,
            xticklabels={$a_1$,$a_0$},
            axis on top,
          ]         
	  \addplot [penColor2!40!background,very thick] {86*(x-7) + 105};
          \addplot [textColor!40!background,dashed] plot coordinates {(7,0) (7,105)};
          \addplot [penColor2,very thick] {49*x-259.4};
          \addplot [textColor,dashed] plot coordinates {(5.78,0) (5.78,23.57)};
          \addplot [very thick,penColor, smooth,samples=100] {x^3 - 4*x^2 - 5*x - 7};
          \addplot[color=penColor3,fill=penColor3,only marks,mark=*] coordinates{(5.78,23.57)};  %% closed hole          
        \end{axis}
\end{tikzpicture}
\caption{Here we see our second guess, along with the linear
  approximation at that point.}
\label{figure:newtonCubic2}
\end{marginfigure}
See Figure~\ref{figure:newtonCubic2} to see what is going on
geometrically.  Again, we repeat our procedure letting $a_2$ be our
next guess, note
\[
f(a_2) \approx 2.97,
\]
we are getting much closer to a root of $f(x)$. Looking at the linear
approximation of $f(x)$ at $x=a_2$, we find
\[
\l_2(x) = f'(a_2)(x-a_2) + f(a_2).
\]
Setting $a_3 = a_2 - \frac{f(a_2)}{f'(a_2)}$, $a_3\approx 5.22$. We now
have $\l_2(a_3) = 0$. Checking by evaluating $f(x)$ at $a_3$, we find
\[
f(a_3) \approx 0.14.
\]

\begin{marginfigure}[0in]
\begin{tikzpicture}
	\begin{axis}[
            xmin=4, xmax=7.5,ymin=-40,ymax=125,domain=(4:8),
            axis lines =center, xlabel=$x$, ylabel=$y$,
            every axis y label/.style={at=(current axis.above origin),anchor=south},
            every axis x label/.style={at=(current axis.right of origin),anchor=west},
            xtick={5.22,5.78,7}, ytickmin=1,ytickmax=0,
            xticklabels={$a_2$,$a_1$,$a_0$},
            axis on top,
          ]         
	  \addplot [penColor2!40!background,very thick] {86*(x-7) + 105};
          \addplot [textColor!40!background,dashed] plot coordinates {(7,0) (7,105)};

          \addplot [penColor2!40!,very thick] {49*x-259.4};
          \addplot [textColor!40!,dashed] plot coordinates {(5.78,0) (5.78,23.57)};

          \addplot [penColor2,very thick] {36.8*x-192.2};
          \addplot [textColor,dashed] plot coordinates {(5.22,0) (5.22,.14)};

          \addplot [very thick,penColor, smooth,samples=100] {x^3 - 4*x^2 - 5*x - 7};
          \addplot[color=penColor3,fill=penColor3,only marks,mark=*] coordinates{(5.22,.14)};  %% closed hole          
        \end{axis}
\end{tikzpicture}
\caption{Here we see our third guess, along with the linear
  approximation at that point.}
\label{figure:newtonCubic3}
\end{marginfigure}
We are now very close to a root of $f(x)$, see
Figure~\ref{figure:newtonCubic3}. This process, Newton's Method, could
be repeated indefinitely to obtain closer and closer approximations to
a root of $f(x)$.


\begin{example}
Use Newton's Method to approximate the solution to 
\[
x^3= 50
\]
to two decimal places. 
\end{example}

\begin{solution} 
To start, set $f(x) = x^3 - 50$. We will use Newton's Method to
approximate a solution to the equation
\[
f(x) = x^3-50 = 0. 
\]
Let's choose $a_0=4$ as our first guess. Now compute 
\[
f'(x) = 3x^2.
\]
At this point we can make a table:
\[
\begin{tchart}{llll}
n &  a_n  & f(a_n)         & a_n - f(a_n)/f'(a_n)\\ \hline
0 & 4    & 14             & \approx 3.708 \\
1 & 3.708 & \approx 0.982  & \approx 3.684 \\
2 & 3.684 & \approx -0.001 & \approx 3.684
\end{tchart}
\]
Hence after only two iterations, we have the solution to three (and
hence two) decimal places.
\end{solution}

In practice, which is to say, if you need to approximate a value in
the course of designing a bridge or a building or an airframe, you
will need to have some confidence that the approximation you settle on
is accurate enough. As a rule of thumb, once a certain number of
decimal places stop changing from one approximation to the next it is
likely that those decimal places are correct. Still, this may not be
enough assurance, in which case we can test the result for accuracy.


Sometimes questions involving Newton's Method do not mention an
equation that needs to be solved. Here you must reinterpret the
question as one that is asking for a solution to an equation of the
form $f(x) = 0$.

\begin{example}
Use Newton's Method to approximate $\sqrt[3]{50}$ to two decimal
places.
\end{example}

\begin{solution}
The $\sqrt[3]{50}$ is simply a solution to the equation
\[
x^3 -50 = 0.
\]
Since we did this in the previous example, we have found
$\sqrt[3]{50}\approx 3.68$.
\end{solution}

\begin{warning}
Sometimes a bad choice for $a_0$ will not lead to a root. Consider 
\[
f(x) = x^3 - 3 x^2 - x - 4.
\]
If we choose our initial guess to be $a_0=1$ and make a table we find:
\[
\begin{tchart}{llll}
n &  a_n  & f(a_n)         & a_n - f(a_n)/f'(a_n)\\ \hline
0 & 1    & -7             & -0.75 \\
1 & -0.75 & \approx -5.359  & \approx 0.283 \\
2 & 0.283 & \approx -4.501 & \approx -1.548 \\
3 & -1.548 & \approx -13.350 & \approx -0.685 \\
4 & -0.685 & \approx -5.044 & \approx 0.432 \\
  \hdotsfor{4}
\end{tchart}
\]
As you can see, we are not converging to a root, which is
approximately $x=3.589$.
\end{warning}



Iterative procedures like Newton's method are well suited for
computers. It enables us to solve equations that are otherwise
impossible to solve through symbolic methods.






